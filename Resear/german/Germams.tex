%_ **************************************************************************
%_ * The TeX source for AMS journal articles is the publisher's TeX code    *
%_ * which may contain special commands defined for the AMS production      *
%_ * environment.  Therefore, it may not be possible to process these files *
%_ * through TeX without errors.  To display a typeset version of a journal *
%_ * article easily, we suggest that you retrieve the article in DVI,       *
%_ * PostScript, or PDF format.                                             *
%_ **************************************************************************
%%
%
%      article type:  book review
%  book information:  \it Free Lie algebras\/}, by Christopher Reutenauer.
%                     Clarendon  Press, Oxford, 1993, xvii+269 pp., \$86.00.
%          tex type:  amstex
%     old file name:  bull589.tex
%     new file name:  199507005.tex.html
%%
\input amstex %version 1.1
\documentstyle{bull}
\keyedby{bull589/mhm}
%\bookrev

\define\dsqcup{\hbox{\mathsurround=0pt$\sqcup\kern-2pt\sqcup$}}
\define\NS{\dsqcup} %there is a new symbol to come for this 
\define\conc{\operatorname{conc}}
\define\END{\operatorname{End}}
\define\Mod{\operatorname{mod}}
\define\GL{\operatorname{GL}}

\topmatter

\revtop     {\it Free Lie algebras\/}, by  Christopher Reutenauer. Clarendon 
Press, Oxford, 1993, xvii+269 pp., \$86.00. ISBN 0-19-853679-8 \endrevtop
\reviewer   G\'erard Duchamp \endreviewer
\affil      Universit\'e de Rouen \endaffil
\ml ged\@lir.dir.univ-rouen.fr \endml 
\endtopmatter

\document               

A {\it free Lie algebra\/} is a Lie algebra freely generated by a set called
{\it alphabet\/}.  The theory of free Lie algebras grew from the need of
computing formally with commutators $[a,b]=ab-ba$ (using only the Jacobi
identity and antisymmetry).  It can be traced back to the turn of the century
in the work of Campbell (1898), Poincar\'e (1899), Baker (1904), and Hausdorff 
(1906).

Given a set $X$ (alphabet) and a ring $K$ of coefficients, we can look at the
immediate ``structural relatives'' of a free Lie algebra which are
\roster
\item "(i)" The free Lie algebra $L(X)$ (or $L_K(X))$
\item "(ii)" The free magma $M(X)$
\endroster

The book reviewed here consists of nine chapters (plus an introductory Chapter
0) presenting the theory and its connections with
\roster
\item "$\bullet$" Coding theory (for variable length codes)
\item "$\bullet$" Control theory 
\item "$\bullet$" Hopf algebras
\item "$\bullet$" Lie groups (exponentials, Hausdorff and Zassenhaus formulae)
\item "$\bullet$" Symmetric group algebra and character theory
\item "$\bullet$" Symmetric and quasisymmetric functions
\endroster

In Chapter 1, the Hopf algebra $K\langle X\rangle=\scr U(L(X))$ is presented. 
Four criteria are described for a noncommutative polynomial to be a Lie 
polynomial.  Among them, one has, in particular, primitivity (known as
Friedrich's criterion) and invariance by Dynkin's projection.  A second law,
the {\it shuffle product\/}, is defined.  This law, denoted  $\NS$, is dual to
the canonical coproduct of $\scr U(L(X))$. Combinatorially, it is the sum of
the words $w$ in which $u$ and $v$ are complementary subwords (this
combinatorial characterization amounts to the etymology for the word {\it
shuffle\/}).

Denoting by {\it conc\/} the {\it concatenation \/} of words and by $c$ the 
comultiplication  of $\scr U(L(X))$, we see that $(K\langle X\rangle,\conc,c,1,
\varepsilon)$ is now a bialgebra (in fact, like every enveloping algebra, it is
a Hopf algebra, but the full structure is not needed here) and $\END(K
\langle X\rangle)$ is an algebra with the {\it convolution law\/} given by
$$f*g(x)=\conc (f\otimes g)(c(x))\.$$

The iterated integrals
$$\int^b_a\,dw$$
are defined. They will occur again in Chapter 6. The fact that the Chen series
$$\sum_{w\in A^*}\left(\int^b_a \,dw\right)w$$
is a grouplike element is proved here.

A direct sum decomposition of $K\langle A\rangle$ which will appear several
times in the book is defined here by means of an $n$-ary operation, the
symmetrized product
$$(P_1P_2\cdots P_n) =\frac 1{n!}\sum_{\sigma\in \germ S_n} P_{\sigma(1)}
P_{\sigma(2)}\cdots P_{\sigma(n)}\.$$
This permits one to give canonical complements of the increasing filtration of
the enveloping algebra of $K\langle X\rangle =\scr U(L(X))$.  More precisely,
let
$$\scr U^n=1+L(X)+L(X)L(X)+\undersetbrace{n\text{ times}}\to{L(X)L(X)\cdots
L(X)}$$
(that is, polynomials which are the products of at most $n$ Lie polynomials),
and define $U_n$ to be the linear span of the products $(P_1P_2\cdots P_n)$.
Then
$$\scr U^{n-1}\oplus U_n=\scr U^n\.$$
It follows that $K\langle X\rangle=\oplus_{n\ge 0}U_n$.  Projections
$$\pi_n: K\langle X\rangle\to U_n$$
corresponding to this special decomposition are called canonical.  It turns out
that each projection will be identified with special elements of $K\germ S_n$. 
In particular, $\pi_1$ allows one to express the coefficients of the Hausdorff
series
$$H(a_1a_2\cdots a_p)=\log (e^{a_1}e^{a_2}\cdots e^{a_p})\.$$

Chapters 4 and 5 are devoted to the construction and study of special (linear)
bases of the free Lie algebra, namely, {\it Hall bases\/}.

The foliages (words obtained from the trees by the natural epimorphism $M(X)\to
X^*)$ of the trees of a Hall set (which are a generalization of classical Hall
sets) form a totally ordered set of words (referred to, therefore, as {\it Hall
words\/}). This set has the complete factorization property; i.e., each word $w$
has a unique decreasing factorization:
$$w=h_1h_2\cdots h_n, \quad h_1\ge h_2\ge \cdots \ge h_n\.$$




Unlike the case of a free group, in a  free monoid not all subsets generate a
free substructure.  A set which is a basis of a free submonoid (for example,
$z(A-z)^*$, the words beginning with their unique occurrence of a given letter
$z)$, is called a {\it code\/}.  Hall sets can be used to construct ``variable
length'' codes with interesting synchronization properties (synchronizing word,
comma-free property). The connection between Hall sets and Lazard's elimination
process is described in the appendix.

The shuffle product endows $K\langle A\rangle$ with the structure of a
commutative associative algebra.  In fact, $(K\langle A\rangle,\NS)$ is an 
algebra of polynomials with the \linebreak  Lyndon words as a transcendence basis.  This,
together with the study of sub-\linebreak  word functions and the combinatorics of subwords,
is the subject of \linebreak  Chapter 6.

We have
$$\mu(w)=\sum_{v\in A^*} \binom wv v$$
where $\mu:F(A)\to K\langle \langle A\rangle\rangle$ is Magnus's transformation
defined on the letters by $\mu(a)=1+a$. The combinatorial coefficients $\binom
wv$ generalize the binomial ones, since $\binom{a^n}{a^p}$ is just the
classical $\binom np$.  For fixed $u\in X^*$, the subword function 
$$\varphi:g\to \binom gu$$
is all representatives (a function $\varphi:G\to K$ is called {\it
representative\/} iff its shifts---i.e., the functions $\varphi_h:x\to
\varphi(hx)$\<---span a finite-dimensional space).

Now comes, in the last two chapters, the representation theoretic aspect of the 
theory.  Since a commutator defines a multilinear operation, there is a natural 
representation of $\GL(V)$ $(V=KA)$ on $L(A)$.  Chapter 8 is devoted to the
action of the symmetric group $\germ S_A\subset \GL(V)$ on $L(A)$: its {\it 
characteristic\/} is computed (that is, in the Schur-Weil duality, the
symmetric function which is the character of the corresponding representation). 
The splitting into irreducibles is given via the {\it major index\/}, a
statistic on Young tableaux, which arises naturally in the representation on
{\it covariants\/} (i.e., the ring of all polynomials factored by the ideal
generated by the constantless symmetric functions).  The computation of the
character of this representation is given.

The symmetric group $\germ S_n$ acts also on words of length $n$ by {\it place
permutation\/}.  Explicitly,
$$(a_1a_2\cdots a_n)\sigma=a_{\sigma(1)}a_{\sigma(2)}\cdots a_{\sigma(n)}\.$$

An idempotent $e\in K\germ S_n$ then defines a projection
$$K_n\langle A\rangle \to K_n\langle A\rangle e,$$
and Lie idempotents are those such that $K_n\langle A\rangle e=L_n(A)$. There
are several known Lie idempotents bearing the names of Dynkin, Klyachko, and
also $\pi_1$ as described earlier. (Recently, Gel\cprime fand et al.\ obtained
all known idempotents from a one-parameter family of Lie idempotents \cite
{1}.)  All these  idempotents (except the orthogonal idempotent discussed in
the appendix) belong to a solvable subalgebra of $K\germ S_n$, which is the
subject of the last chapter---Solomon's descent algebra.

A {\it descent\/} of a permutation $\sigma\in\germ S_n$ is an index $1\le i\mu} \pi_\mu
\Gamma_n\pi_\lambda)$$
where $\ge$ is an order on {\it partitions\/} (decreasing compositions) such
that $\pi_\mu\pi_\lambda\ne 0 \Rightarrow \lambda\ge \mu$.  Note that in \cite
{1} another realization of the algebra $\oplus\Sigma_n$ is given in terms of {\it
noncommutative symmetric functions\/}.

\end{document}

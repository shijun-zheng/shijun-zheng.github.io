\documentclass[12pt]{article}
\usepackage{amsmath}
\usepackage{amsthm}
\usepackage{amsfonts}
\usepackage{amssymb}
\usepackage{latexsym}

\usepackage[mathscr]{eucal}
\usepackage{eucal} %(with mathscr option; Euler script alphabet)
%\usepackage{eufrak} 
%\renewcommand{\Re}{{\mbox{Re}}}

%\def\eps{\varepsilon}
\def\vphi{\varphi}
\def\lam{\lambda}
\def\Om{\Omega}
\def\de{\delta}
%\def\sh{\sinh}
%\def\ch{\cosh}
%\def\th{\tanh}
%\def\F{{\cal F}}
\newcommand{\scr}{\mathscr}
\newcommand{\ud}{\mathrm{d}}

\newtheorem{theorem}[subsection]{Theorem}
\newtheorem{lemma}[subsection]{Lemma}
\newtheorem{corollary}[subsection]{Corollary}

\newtheorem{remarks}[subsection]{Remarks}
\newtheorem{definition}[subsection]{Definition}

%\include{psfig}

\title{Multipliers of Hardy spaces on local fields}
\author{Shijun Zheng \\
        {\normalsize {\sl Department of Mathematics }} \\
        {\normalsize {\sl University of Maryland }}\\
        {\normalsize {\sl College Park, MD 20742}}\\
        {\normalsize {\sl USA}}\\
}
%and\\      
%{}\\
%    {\normalsize {\sl Department of Mathematics }} \\
%        {\normalsize {\sl Nanjing University}}\\
%        {\normalsize {\sl Nanjing, 210093 }}\\
%        {\normalsize {\sl China, P.R.}}\\   
        
%\date{}

\begin{document}
\maketitle
%\tableofcontents


\begin{abstract}

 In Section 3, a simple induction enables us to prove such
extensions on $K^n$, the $n$-dimensional linear space over a local field $K$,  

\end{abstract}


\section{Introduction}

Preliminary.  $K$ locally compact, totally disconnected, nondiscrete
(complete) field.  If $char\; K=p$, it is a field of formal Laurent
series over $GF(q)$, the Galois field, $q=p^c $. If $char\; K=0 $, then
$K$  is either a $p$-adic field or a finite algebraic extension of
such a field.  Every $x\in {\cal P}^s:=\{ |x|\le q^{-s} \} =\ss^s {\cal O}$.
$\ss$ is the prime (irreducible ) element in $K$. $\{ {\cal P}^s
\}_{s\in \mathbb{Z}}$ neighborhood system in $K$.
 
Every $x$ has a unique representation
$$
x=\sum_{\ell=s }^\infty x_\ell \ss^\ell; 
$$
norm:  $|x|= q^{-s}$ if $x_s\neq 0$. With this norm $K$ is a space of
homogeneous type in the sense of Coifman and Weiss. We can define
$H^p$ spaces via atomic decomposition.

\section{Multiplier on Hardy spaces over $K$}
 
A multiplier theorem of Marcinkiewicz type.

\begin{theorem} Suppose $\phi$ is radial.  $\{\phi\}\in BV(\mathbb{Z})$,
i.e.,
\begin{equation}
\sum_{-\infty}^\infty \vert \phi^k-\phi^{k+1}\vert <\infty. 
\end{equation} 
Then ${\phi}^\vee \in L^1(K) +\phi(0) \de $. Moreover, 
$\Phi^*f (x)=\sup_k |\phi_k(D)f(x) |$   has type $(H^p, L^p)$, $0<p\le 1$.  
\end{theorem}

In [ZhZ], we prove a multiplier theorem for maximal operators on
$L^p$.    

\begin{theorem}
Suppose $\Om\in L^\infty$ satisfies
\begin{equation}
(i)\quad \sup_{\substack{|z|=1\\k\in N\\ \nu \in Z} }\sum_{m=1}^k
\int_{|x|=1}\vert \sum_{\ell=m+1}^{k+1}  [ ( \Om^{\ell,\nu}
)^\vee(x+\ss^mz)-( \Om^{\ell,\nu}  )^\vee(x)   ]  \vert \ud x
:=B<\infty,
\end{equation}
where $( \Om^{\ell,\nu} = ( \Om(\ss^\nu\cdot)\Phi^\ell  $; $B>0$ a
constant. Then $\Om \in SMM(L^p)$, with regularization, $1<p<\infty$;
and, $Lf= \lim_{k\rightarrow -\infty } L_kf  $ in $L^p$ and $a.e.$ in
$x$ for $ f\in L^p$, $ 1<p< \infty $.

Furthermore, if the stronger condition
\begin{equation}
(ii)\quad \sup_{\substack{|z|=1\\ \nu \in Z}} \sum_{\ell=2}^\infty \sum_{m=1}^{\ell-1}
\int_{|x|=1}\vert [ ( \Om^{\ell,\nu}
)^\vee(x+\ss^m z)-( \Om^{\ell,\nu}  )^\vee(x)] \vert \ud x
:=B'<\infty,
\end{equation}
is satisfied, then $ \Om \in SMM(L^p)\cap WMM(L^1)$ with
regularization, $1<p<\infty$ and
 $Lf= \lim_{k\rightarrow -\infty } L_kf$ exists  $a.e.$ in
$x$ for $ f\in L^p$, $ 1\le p< \infty $. 
\end{theorem}

Remark. (ii) implies (i).


When $\Om$ is homogeneous of degree zero, i.e., $\Om(\ss^k x )=\Om(x)$
for all $k\in Z$. Then Theorem 1 can be restated in a simpler form.

\begin{theorem} (Theorem 2)
Suppose $\Om\in L^\infty$ is homogeneous of degree zero.

(i)\quad If
$$
 \sup_{|z|=1} \sum_{m=1}^k
\int_{|x|=1}\vert \sum_{\ell=m+1}^{k+1} [(\Om^{\ell}
)^\vee(x+\ss^mz)-( \Om^{\ell} )^\vee(x)] \vert \ud x
:=B<\infty,
$$
then the first conclusion of Theorem 1 holds.

(ii) If 
$$
\quad \sup_{\substack{|z|=1}} \sum_{\ell=2}^\infty \sum_{m=1}^{\ell-1}
\int_{|x|=1}\vert [ ( \Om^{\ell}
)^\vee(x+\ss^m z)-( \Om^{\ell} )^\vee(x)] \vert \ud x
:=B'<\infty,
$$
then the second conclusion of Theorem 1 holds.
\end{theorem}


We also have a result on radial multiplers associated with dilation
\begin{theorem}
Suppose that $\Om$ satisfy all the conditions in Theorem 2 and $m$ is
radial and
satisfies  $\sum_{k=-\infty}^\infty |m^k-m^{k+1} | <\infty $,
Let 
$$
T_kf(x)=[ (m\Om)(\ss^{-k}\cdot ) \hat{f}]^\vee(x)
$$
be defined initially for $f\in {\cal S}$, and let 
$$
T^*f(x)= \sup_{k\in Z} \vert T_kf(x)\vert.
$$

Then 

$
(i) \lim_{k\rightarrow -\infty} T_kf(x)=m(0) (\Om  \hat{f})^\vee(x)
\quad in \quad L^p, 1<p<\infty;
$

$ \lim_{k\rightarrow -\infty} T_kf(x)=m(0) (\Om  \hat{f})^\vee(x)$
\quad a.e. in $x$, $1\le p<\infty$, where $m(0)=  \lim_{k\rightarrow
-\infty} m^k; 
$

$ T^*f$ has type $(p, p)$, $1<p<\infty$ and weak type $(1,1)$; in brief,
$m\Om$ belongs to $SMM(L^p)\cap WMM(L^1)$,  $1<p<\infty$ with dialation.
\end{theorem}

\vspace{.421in}

\section{Some References}

Daly, James E.(1-CO2) 
A necessary condition for Calder�n-Zygmund singular integral operators. (English. English summary) 
J. Fourier Anal. Appl. 5 (1999), no. 4, 303--308.

Let $\Omega(x) $ be an integrable function defined on the sphere $S^{n-1}$ with mean value zero and let $T_\Omega$ be the Calder�n-Zygmund singular integral
operator with kernel the principal value distribution $\text{p.v.}\,\Omega (x/|x|) /|x|^n$. The following result is proved: If $T_\Omega$ maps the Hardy space $H^1(\bold
R^n)$ into itself, then the function $\Omega$ is an element of the Hardy space $H^1(S^{n-1})$. Corollaries are obtained regarding summability properties of the $L^2$
norms of the elements of the spherical harmonic decomposition of the corresponding multipliers. The proof of the main result uses the spherical harmonic expansion of
the corresponding multiplier on $S^{n-1}$, a version of the F. and M. Riesz theorem on the sphere, and Colzani's characterization of the Hardy space $H^1$ on the
sphere using the Riesz transforms.

                                    Reviewed by Loukas Grafakos 
---------------------------------------------------------------

Daly, James E. 
The invalidity of the Calder�n-Zygmund inequality for singular integrals over local fields. 
Bull. Amer. Math. Soc. 81 (1975), no. 5, 896--899.

The author continues his study in analysis over local fields [see Math. Ann. 215 (1975), 59--68; ibid. 218 (1975), 117--126]. Let $K$ be a local field
(non-discrete, zero-dimensional, locally compact field), $B^n=\{x\in K\colon|x|\leq q^{-n}\}$, $D^n=\{x\in K\colon|x|=q^{-n}\}$, $\lambda$ a Haar measure on $K$ such
that $\lambda(B^0)=1$, and $\pi$ a prime satisfying the condition $\pi B^0=B^1$. For each integer $k\geq 2$, let $\Gamma_k$ denote the set of $\omega\colon
K^*\rightarrow C$ such that (i) $\omega(x+B^k)=\omega(x)$ for each $x\in D^0$, (ii) $\omega(\pi^jx)=\omega(x)$ for any $x\in K^*$ and $j\in Z$, and (iii)
$\int_{D^0}\omega(x)\,d\lambda(x)=0$. Set $\Gamma=\bigcup_{k=2}^\infty\Gamma_k$. For each $\omega\in\Gamma$, we define the singular integral operator
$T_\omega$ with kernel $\omega$ by setting $T_\omega(f)(y)=\text{p.v.}\int_K\omega(x)|x|^{-1}f(y-x)\,d\lambda(x)$; then $T_\omega$ defines a bounded linear operator
in each $L^p$, $1\leq p<\infty$. We also set $$ \|\omega\|_r=\{\int_{D^0}|\omega(x)|^r\,d\lambda(x)\}^{1/r}. $$ The author proves the following theorem. For $1<p<\infty$,
$1\leq r<\infty$ ($1\leq r\leq\infty$, if $q$ is odd), there is no constant $C(p,r)$ such that for all $\omega\in\Gamma$, $\|T_\omega\|_p\leq C(p,r)\|\omega\|_r$. This means
that the Calder�n-Zygmund inequality for singular integrals over $ R^n$ [cf. N. Dunford and J. Schwartz, Linear operators, Part II: Spectral theory, self adjoint operators in Hilbert
space, especially p. 1072, Interscience, New York, 1963; Russian translation, Izdat. "Mir", Moscow, 1966] is no longer valid in the case
of local field analysis.

                                     Reviewed by M. Hasumi 

-----------------------------------------------------------

Daly, James E.(1-CO2) 
Factorization of Hardy spaces on local fields. 
Math. Ann. 282 (1988), no. 2, 243--250.

Coifman, Rochberg and Weiss extended to $\mathscr{ H}'$ of several
variables 
the factorization theorems on the unit disk involving conjugate functions. Uchiyama
generalized the work to $\mathscr{ H}^p$ on spaces of homogeneous type for a restricted range of $p$. The author extends these results to $\mathscr{ H}^p$ on a local field for all
$p$. A local field $\mathfrak{ k}$ is a field with nontrivial
valuation which is locally compact. When not
 equal to the real or complex numbers, $\mathfrak{ k}$ is a complete
non-Archimedean valued field $(|a+b|\le \max |a|, |b|$ for all
$a,b\in\mathfrak{ k})$ with 
finite residue class field and discrete valuation $(|\mathfrak{ k}|=\{q^n\colon n\in \bold Z\}
\cup\{0\})$. The author proves a number of interesting results pertaining to singular integrals $T_w$ with a kernel $w$ that is homogeneous and locally constant. Among
them is the following theorem: If $g\in \mathscr{ H}^q$, $h\in \scr{ H}^r$ with $1/p=1/q+1/r$, $r,p,q>0$ then there is a constant $C(q,r)$ such that $||hT_w g-g T'_wh||_{\mathscr
{H}^p}\le C(q,r)||g||_{\scr{ H}^q} ||h||_{\mathscr{ H}^r}$. The
singular integrals $T_w$ and $T'_w$ are defined 
by $$ T_wf(y)=\textrm{pv}\int(w(y-x)/|y-x|)f(x)dx,$$ 
$$
T'_wf(y)=\textrm{pv}\int(w(x-y)/|x-y|) f(x)dx.$$ They were studied
 extensively by the author and 
 K. Phillips \ref[same journal 265 (1983), no. 2, 181--219] and 
 M. Taibleson \ref[ Fourier analysis on local fields, Princeton Univ. Press, Princeton, NJ, 1975].

                                    Reviewed by E. Beckenstein 
-------------------------------------------------------------------

Daly, James E.(1-CO2); Phillips, Keith(1-NMS) 
On the classification of homogeneous multipliers bounded on $H\sp 1( R\sp 2)$. 
Proc. Amer. Math. Soc. 106 (1989), no. 3, 685--696.

Suppose that $m$ is a bounded measurable function on $\bold R^2$, homogeneous of degree $0$, and that the corresponding multiplier operator $T_m$ is bounded on
$H^1(\bold R^2)$. The authors show that if $m$ is regarded as a function on the circle $\bold T$, then the convolutions $S*m$,\break $S*(\sin (\cdot)m)$, and $S*(\cos(\cdot)m)$
are of bounded variation on $\bold T$, where $S$ is determined by $\hat S(n)=2\pi(-i\,\textrm{sgn}(n))^{n+1}$, from which it follows in turn that $m$ has an absolutely
convergent Fourier series. The proof makes use of detailed analysis involving special functions. 

A simple example shows that $T_m$ may fail to be bounded on $H^1$ even though $m$ has an absolutely convergent Fourier series. The authors do show, however,
using the method of rotations and Riesz transforms, that if the three convolutions above are of bounded variation on $\bold T$, then $T_m$ is bounded on $L^p(\bold
R^2)$ for $1<p<\infty$.

                                    Reviewed by Norman J. Weiss 
-------------------------------------------------------------------

Daly, James E. 
Local field singular integrals with complex homogeneity. 
Math. Ann. 218 (1975), no. 2, 117--126.

The authors study singular integral operators in several variables over a local field $\scr K$ where the kernel has complex homogeneity. These operators are of the
form $T_\omega{}^\gamma f(y)=(\text S)\int(\omega(x)/|x|^{\alpha+i\gamma})f(y-x)\,dx$, where $(\text S)\int$ denotes Abel summation, $\gamma\in R\smallsetminus\{2\pi
n/\log q\}_{n=-\infty}^\infty$ and $\omega$ is homogeneous of degree zero and satisfies certain smoothness conditions. Their results are closely related to the zero
homogeneity work of K. Philips and M. Taibleson [Pacific J. Math. 30 (1969), 209--231] and the fractional integration work of Taibleson [Math. Ann. 176
(1968), 191--207].

                                     Reviewed by U. B. Tewari 

-------------------------------------------------------------




Daly, James E. 
On the necessity of the H�rmander condition for multipliers on $ H\sp{p}( R\sp{n})$. 
Proc. Amer. Math. Soc. 88 (1983), no. 2, 321--325.

From the text: "We prove that a class of multiplier operators on $ H\sp p( R\sp n)$, that send atoms to molecules boundedly, must satisfy a Hormander condition. This
provides a partial converse to a theorem of Taibleson and Weiss."


----------------------------------------------------------------------------------
Daly, James E. 
Singular integral multipliers on local fields. 
Harmonic analysis in Euclidean spaces (Proc. Sympos. Pure Math., Williams Coll., Williamstown, Mass., 1978), Part 2, pp. 333--335, 
Proc. Sympos. Pure Math., XXXV, Part, 
Amer. Math. Soc., Providence, R.I., 1979. 

Let $K$ be a local field with norm $|�|$. For a function $m$ on $K$, which is (integrally) homogeneous of degree zero (i.e., $f(\pi^kx)=f(x)$ for all $x\in K$, $k\in Z$ and
some $\pi\in K$ with $|\pi|=q^{-1}$), let $m^0$ be the restriction of $m$ to $D^0=\{x\in K\colon|x|=1\}$. The author presents some results on the multiplier operator $T_m$
associated with such a function $m$ in $L_\infty(K)$. Typical are the
following two theorems. Theorem A: If
 $\sum_{n=0}^\infty|(m^0)^{\small \vee}(n)|<\infty$ then there
exists an $\omega\in L_1(D^0)$ such that $T_m$ equals the singular
integral operator associated with $\omega$. 

Theorem B: If $\sum_{n=2}^\infty(\log n)|(m^0)^{\small
\vee}(n)|<\infty$ then $T_m$ maps $H_1(K)$ continuously into $H_1(K)$. 

                                    Reviewed by C. W. Onneweer 
--------------------------------------------------------------------

Daly, James E.(1-CO2); Phillips, Keith L.(1-CO2) 
Multipliers and square functions for $H\sp p$ spaces over Vilenkin groups. (English. English summary) 
Analysis of divergence (Orono, ME, 1997), 171--186, 
Appl. Numer. Harmon. Anal., 
Birkh�user Boston, Boston, MA, 1999. 

Summary: "This work is a short history of the Hardy spaces $H^p$ for Vilenkin groups, beginning with the square function characterizations of the Lebesgue spaces
$L^p$ by Sunouchi in the dyadic case in 1951. For the dyadic group $G$ it is proved that the function with values $\phi(k)=k2^{-j}$ on the $j$th dyadic block of $G$ is
an $H^p$-multiplier with respect to the Walsh functions $\{\omega_k\colon 0\leq k\leq\infty\}$. This theorem implies that Sunouchi's square function $S_2$
characterizes $H^p$, solving a conjecture of Simon for $H^1$ made in 1985. The inverse of $\phi$ is also an $H^p$-multiplier." 

------------------------------------

Daly, James; Phillips, Keith(1-NMS) 
On singular integrals, multipliers, $\mathfrak{ H}\sp{1}$ and Fourier series---a local field phenomenon. 
Math. Ann. 265 (1983), no. 2, 181--219.

This article extends earlier works of M. H. Taibleson [especially Fourier analysis on local fields, Princeton Univ. Press, Princeton, N.J., 1975] and the
authors [especially Phillips, Math. Ann. 242 (1979), no. 1, 69--84; MR 81e:43018]. Harmonic analysis on a local field $K$ other than $ R$ or $ C$ allows one to use in
this context the notions of singular integral, Hardy spaces and their Coifman-Weiss molecular characterizations [see Taibleson, Harmonic analysis in Euclidean
spaces (Williamstown, Mass., 1978), Part 2, 311--316, Proc. Sympos. Pure Math., XXXV, Part 2, Amer. Math. Soc., Providence, R.I., 1979; MR 81i:43015]. Let $P\sb
0$ be the ring of integers and $D$ the group of units of $K$. The distributions on $K$ [resp. $P\sb 0$, $D$] are the linear functionals on the class of locally constant test
functions with compact support. Let $w$ be a distribution on $D$ with total mass 0. Convolution with respect to the kernel $(1/\vert x\vert )w(x/\vert x\vert )$ defines an
operator $T$ on the test functions on $K$. Applying the Fourier
transform on $K$, one then has
 $\hat{(T\varphi)} =m�\hat\varphi$, where $m$ is a distribution on
$D$. A distribution $u$ on $D$ can also be considered to be a distribution on $P\sb 0$ with support contained in $D$. The connection between the Fourier transforms
$u\sp \Delta$ on $D$ and $u\sp {\dag}$ on $P\sb 0$ involve the gamma function on $K$, for which numerous formulas are established. From the relations between
$w\sp \Delta$, $w\sp {\dag}$, $m\sp \Delta$, $m\sp {\dag}$, the authors establish links between the membership of these functions in the spaces $c\sb 0$, $l\sp p$,
$1\leq p<+ \infty$, and the membership of $w$ or $m$ in $L\sp p(D)$. For example, $m\sp {\dag}\in l\sp 1$ implies $w\in L\sp 1(D)$. Contrary to the situation in $ R\sp
n$, an example is given in which $w\in L\sp 2(D)$ and $m\notin L\sp
\infty(D)$ (a result established earlier by J. A. Chao [Studia
Math. 49 (1973/74), 268--287). 
Let ${\cal B}(L\sp p)$, ${\cal B}(H\sp p)$, and ${\cal B}(H\sp p,L\sp
p)$ be the spaces of bounded operators in 
$L\sp p(K)$, in $H\sp p(K)$, and from $H\sp
p(K)$ into $L\sp p(K)$, respectively. The membership of $T\sb m$ in one of these operator spaces depends on suitable properties of $w$ or $m$. For example, if $T\sb
m\in{\cal B}(H\sp p,L\sp p)$ for $0<p\leq 1$, then $m\sp {\dag}\in l\sp p$. Conversely, if $m\sp {\dag}$ satisfies the stronger property $\sum\sb {n\geq 1}\vert m\sp
{\dag}(\chi\sp {(n)})\vert \sp p\log n<+\infty$ for a suitable indexing of the character group of $P\sb 0$, then $T\sb m\in{\cal B}(H\sp p,L\sp p)$. Other conditions involve
the continuity of $w$: Let $w\in L\sp 1(D)$, $\tilde w(x)=(1/\vert x\vert )w(x/\vert x\vert )$ and $$\eta(w)=\int\sb {\vert y\vert \leq 1}\int\sb {\vert x\vert >1}\vert \tilde
w(x-y)-\tilde w(x)\vert \,dx\,dy.$$ If $\eta(w)>+\infty$, then $m\in L\sp \infty(D)$, $m\sp {\dag}\in l\sp 1$, $T\sb m\in{\cal B}(L\sp P)$ for $1<p<+\infty$, and $T\sb
m\in{\cal B}(H\sp 1,L\sp 1)$. Various conditions implying $\eta(w)<+\infty$ are given, for example, $\sum\sb {n\geq 1}\vert m\sp {\dag}(\chi\sp {(n)})\vert \log n<+\infty$.
The last section is dedicated to examples of Banach algebras ${\cal M}({\cal F})=\{cI+T\sb m\}$, $c\in C$, of operators in which $m$ belongs to a suitable family ${\cal
F}$ of distributions on $D$. One such family ${\cal F}$ is the set of all $m$ such that $m\sp \Delta\in l\sp 1$ and $m\sp \Delta(1)=0$; the norm in ${\cal M}({\cal F})$ is
$\vert c\vert +\Vert m\sp \Delta\Vert \sb 1$. This article constitutes a work of synthesis covering almost all aspects of harmonic analysis on local fields. Many explicit
calculations are presented.

                                   Reviewed by Michel Gatesoupe 
----------------------------------------------------------------

Onneweer, C. W.(1-NM-MS) 
H\"ormander-type multipliers on locally compact Vilenkin groups: the $L\sp 1(G)$-case. (English. English, Russian summary) 
Anal. Math. 24 (1998), no. 3, 213--220.

Let $G$ be a locally compact Vilenkin group. The author establishes a multiplier theorem for the space $L\sp 1(G)$ and discusses its sharpness. As a corollary, he
obtains a H\"ormander-type multiplier theorem for $L\sp 1(G)$.

                                    Reviewed by Da Chun Yang 

-------------------------------------------------------------

Kitada, Toshiyuki(J-HIROSG); Onneweer, C. W.(1-NM-S) 
H�rmander-type multiplier theorems on locally compact Vilenkin groups. (English. English summary) 
Theory and applications of Gibbs derivatives (Kupari-Dubrovnik, 1989), 115--125, 
Mat. Inst., Belgrade, 199?. 

The authors present a discussion of dyadic differentiation on locally compact Vilenkin groups and related H\"ormander-type conditions for multipliers on weighted
Lebesgue and Hardy spaces. Proofs for these results can be found in papers by Kitada \ref[Monatsh. Math. 110 (1990), no. 3-4, 283--295; MR 91k:43010] and
Onneweer and T. S. Quek \ref[Proc. Amer. Math. Soc. 105 (1989), no. 3, 622--631; MR 89g:43004; J. Austral. Math. Soc. Ser. A 48 (1990), no. 3, 472--496; MR
91j:43007]. The results are analogous to the real case results by Kurtz, Wheeden, Muckenhoupt, and Young. 

{For the entire collection see MR 93k:42001.}

                                    Reviewed by James E. Daly 

-------------------------------------------------

Onneweer, C. W.(1-NM); Quek, T. S.(SGP-SING) 
Multipliers for Hardy spaces on locally compact Vilenkin groups. (English. English summary) 
J. Austral. Math. Soc. Ser. A 55 (1993), no. 3, 287--301.

Let $G$ be a locally compact Vilenkin group with dual group $\Gamma$, and, for each $0<p\leq 1$, let $H^p$ denote the correspondinng dyadic Hardy space on $G$
(i.e., the space of functions $f$ on $G$ whose dyadic maximal function $f^*$ belongs to $L^p(G)$). An $L^\infty (\Gamma)$ function $\phi$ is called a multiplier of $H^p$
if there is a constant $C>0$ such that the Vilenkin-Fourier transform
of $f$ satisfies 
$\| {(\phi \hat{ f})}^\vee \|_{H^p} \leq C\|f\|_{H^p}$ for all $f\in H^p$. 

The authors obtain several conditions on a given $\phi \in L^\infty (\Gamma)$, using Herz norms, sufficient to conclude that $\phi$ is a multiplier of $H^p$, and discuss
the sharpness of some of these conditions. They also include an illuminating discussion of how these results relate to earlier theorems of Seeger, of Cowling, and of
Fendler and Fournier. The proofs use the atomic characterization of $H^p$.

                                     Reviewed by W. R. Wade 

------------------------------------------------------------

Onneweer, C. W.(1-NM-S) 
Multipliers for $H\sp p(G)$-spaces. (English. English summary) 
X Latin American School of Mathematics (Spanish) (Tanti, 1991). 
Rev. Un. Mat. Argentina 37 (1991), no. 1-2, 135--141 (1992).

Summary: "In this paper we present some multiplier theorems for Hardy spaces defined on a locally compact Vilenkin group. We discuss how these theorems compare
with some known multiplier theorems for Lebesgue and Hardy spaces defined on $\bold R^n$, and we conclude with a brief description of a few open questions." 

-----------------------------------------------------------------

Onneweer, C. W.(1-NM); Quek, T. S.(SGP-SING) 
On $H\sp p(\bold R\sp n)$-multipliers of mixed-norm type. (English. English summary) 
Proc. Amer. Math. Soc. 121 (1994), no. 2, 543--552.

Let $0<p\leq 2$, and $1/s=1/p-1/2$. Let $\eta$ be a suitable bump function associated to the unit annulus in $ R^n$. The following theorem is proved: If $0<p\leq 1$,
$m\in L^\infty( R^n)$ and $\sum_{j\in Z}\|[m(2^j\cdot)\eta(\cdot)]^\vee\|^s_p<\infty$, then $m$ is a Fourier multiplier of $H^p( R^n)$. The authors also show, by example, that
replacing $s$ by a larger number in the statement of the theorem is not sufficient to yield the conclusion. These results are direct analogues of those previously
obtained by the authors for locally compact Vilenkin groups. They also nicely complement a result of M. G. Cowling, G. Fendler and J. J. F. Fournier \ref[Math. Ann.
285 (1989), no. 2, 333--342; MR 90m:42016], which says that if $1<p\leq 2$ and $\sum_{j\in Z}\|m(2^j\cdot)\eta(\cdot)\|^s_{m_p}<\infty$, then $m$ belongs to the space $m_p$
of $L^p( R^n)$ Fourier multipliers.

                                    Reviewed by Anthony Carbery 

---------------------------------------------------------------

Onneweer, C. W.(1-NM-S); Quek, T. S.(SGP-SING) 
$H\sp p$ multiplier results on locally compact Vilenkin groups. 
Quart. J. Math. Oxford Ser. (2) 40 (1989), no. 159, 313--323.

 A. Baernstein and E. T. Sawyer
 \ref[Mem. Amer. Math. Soc. 53 (1985), no. 318; MR 86g:42036] proved strong multiplier theorems that provide sufficient
conditions for a multiplier operator to be bounded on the Hardy spaces
 $H^p(\mathbb{ R}^n)$ for $0<p\leq 1$. 
They showed also that, in a certain sense, their
results are the best possible. 

 T. Kitada provided analogues of the sufficiency results of
Baernstein and Sawyer 
for Hardy spaces defined over certain locally
compact Vilenkin groups. In the paper under review, the authors provide a modification of Kitada's result for $H^1$ and show that their result for $H^1$ and Kitada's for
$H^p$, $0<p<1$, are the best possible. The smoothness conditions on the multipliers are expressed in terms of their containment in certain Herz spaces.

                                    Reviewed by James E. Daly 


---------------------------------------------------------------


\vspace{0.3in}

\begin{thebibliography}{99}

\bibitem{D}
 Daly, James E. A necessary condition for Calder\'on-Zygmund singular
integral operators, {\em  J. Fourier Anal. Appl}. {\bf 5} (1999), no. 4, 303--308. 

\bibitem{DK}
Daly, James E.; Kurtz, Douglas S. BMO and singular integrals over
local fields, {\em J. Austral. Math. Soc. Ser. A} {\bf 54} (1993), no. 3,
321--333. 

\bibitem{DP}
 Daly, James E.; Phillips, Keith L. Multipliers and square functions for $H\sp p$ spaces over Vilenkin groups. Analysis of divergence (Orono, ME, 1997),
  171--186, Appl. Numer. Harmon. Anal., Birkh�user Boston, Boston, MA, 1999. 
 
\bibitem{DP2}
 Daly, James E.; Phillips, Keith L. A note on $H\sp 1$ multipliers for locally compact Vilenkin groups. Canad. Math. Bull. 41 (1998), no. 4, 392--397.

\bibitem{DP3}
Daly, J. E.; Phillips, K. L. Walsh multipliers and square functions for the Hardy space $H\sp 1$. Acta Math. Hungar. 79 (1998), no. 4, 311--327.

\bibitem{DP4}
Daly, James; Phillips, Keith On singular integrals, multipliers, $\mathfrak{ H}\sp{1}$ and Fourier series---a local field phenomenon. Math. Ann. 265
  (1983), no. 2, 181--219. 
 
\bibitem{O}
Onneweer, C. W. H\"ormander-type multipliers on locally compact Vilenkin groups: the $L\sp 1(G)$-case. Anal. Math. 24 (1998), no. 3, 213--220.


\bibitem{OK}
Kitada, Toshiyuki; Onneweer, C. W. H\"ormander-type multiplier theorems on locally compact Vilenkin groups. Theory and applications of Gibbs derivatives
  (Kupari-Dubrovnik, 1989), 115--125, Mat. Inst., Belgrade, 199?. 
                                
\bibitem{OQ}
Onneweer, C. W.; Quek, T. S. Multipliers for Hardy spaces on locally
compact Vilenkin groups,
{\em J. Austral. Math. Soc. Ser.A}, {\bf 55} (1993), no. 3, 287--301.

\bibitem{OQ2} 
Onneweer, C. W.; Quek, T. S. $H\sp p$ multiplier results on locally
compact Vilenkin groups, {\em Quart. J. Math.}, Oxford Ser. (2) {\bf 40} (1989), no. 159,
  313--323. 

\bibitem{T}
Taibleson, M.H., {\em Fourier Analysis on Local Fields}, Princetion
University Press, 1975.

\bibitem{Zhen}
 Riesz type kernels over the ring of integers of a local field, {\em
J. Math. Anal. Appl.}, {\bf 208} (1997), no. 2, 528--552. 

\bibitem{Zhen2}
S.Zheng, Ces\`aro summability of Hardy spaces on the ring of integers in
a local field,
{\em J. Math. Anal. Appl.}, {\bf 249} (2000), no. 2, 626--651. 

\bibitem{Zh-Zh}
S.Zheng,  and W.Zheng, Almost everywhere convergence of sequences of
multiplier operators on local fields, {\em Sci. China Ser. A} {\bf 40} (1997), no. 1,
  10--23.                                


\end{thebibliography}

\end{document} 












 














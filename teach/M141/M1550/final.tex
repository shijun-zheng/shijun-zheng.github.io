\documentclass[12pt]{article}
\usepackage{amsmath,amssymb, amsfonts}
\usepackage{latexsym}
\pagestyle{empty}

\renewcommand{\baselinestretch}{1.2}
\parindent0em

\begin{document}
\begin{large}
\begin{bf}
\hspace{.75in}
{MATH 1550-24, Final Exam} %. Fri. Apr.30, 2004}
\hspace{1in}
\parbox{1in}{ Name \newline Id}
\end{bf}
\end{large}

\vspace*{.152in}

Put the question number (including its sub-problem number, if any) for each 
problem on your answer sheet. 
Put a box around the final answer to a question.$^*$

\vspace*{.02in}
For full credit you must \emph{show your work}. You must have enough
written work, including explanations when called for, to justify your
answers. Incomplete solutions may receive partial credit if you have
written down a reasonable partial solution. %Correct answers may not
                                            %receive credit if
                 

\vspace{.25in}

{\bf 1 [14P])} Write the equation for the tangent line and the
normal line to the graph of 
${\displaystyle f(x) = x^3 - x +2 }$
at the point $P(-1,2)$. \\

Tangent line:\hbox to 3.5cm{\hrulefill}\hfill
Normal line:\hbox to 3.5cm{\hrulefill}\hfill


%b)  $\displaystyle{y = x \cos (x)}$
%at the point $P(\frac{\pi}{2},0)$.

%Tangent line:\hbox to 3.5cm{\hrulefill}\hspace{1.5cm}
%Normal line:\hbox to 3.5cm{\hrulefill}\hfill

\vspace{3.360in}

$\text{-------------------------------------------------}$

\begin{small}
$*$ You can use the back of answer sheet when needed. % but be sure to write the question number.
\end{small}

%\newpage


{\bf 2 [18P])} Find the derivative of the following functions\\

a) ${\displaystyle f(x) = \frac{\tan x}{x}. \quad f^\prime (x) = }$
\\

b) ${\displaystyle f(x) =  5\sqrt{x}+e^{5x}. \quad
f^\prime (x) = }$
%\nlc{2}
\\

c)  ${\displaystyle g(x) = \ln \sqrt{x^2-1}. \quad
g^\prime (x) = }$

\vspace{3.95in}


{\bf 3 [16P])}  Find the limits

a) $\displaystyle{
\lim_{x\to 1}\frac{x^3 +x -2}{x- 1}=
\hbox  to 3cm{\hrulefill}}$ 

\vspace{.5in}

b) ${\displaystyle \lim_{x\to \infty}\frac{2x+7\ln x}{2x+7}=
\hbox to 3cm{\hrulefill}}$

\vspace{.5in}

{\bf 4 [16P])} Find the maximum and minimum values of
the function 

 $\displaystyle{f(x) = \frac{x}{4} + \frac{4}{x}}$ on the
closed interval $\displaystyle{[1,5]}$.
\\
The maximum is:\hbox to 3cm{\hrulefill}\hspace{1.5cm}\hfill
\\
The minimum value is: \hbox to 3cm{\hrulefill}
\\

%b)
%$\displaystyle{f(x) = |x^2-1| +\frac{1}{2}x^2}$ on the
%closed interval $\displaystyle{[0,2]}$.
%\\ %\nl{5}
%The maximum is:\hbox to 2cm{\hrulefill}\hspace{1.5cm}
%The minimum value is: \hbox to 2cm{\hrulefill}

\vspace{3.0in}

{\bf 5 [18P])}  Evaluate the following indefinite integral:
\\ %\nl{5}
a) 
$\displaystyle{\int \left( e^x + x^{2/3}
\right)\, dx =}$
%\nlc{2} %\nl{5}
\\

b) $\displaystyle{\int \frac{\ln (x)}{x} \, dx=}$
\\

c) $\displaystyle{\int \textrm{sec}^2(y+2) \, dy=}$
\\



\vspace{3.25in}

{\bf 6 [24P])} Evaluate the following definite integrals:
\\ %\nl{5}

a)  $\displaystyle{\int_{0}^1 (1 - x)^{1.5}\, dx =}$
\\

b) ${ \int_{-\pi}^{\pi} \sin x \cos x\, dx\, = }$
\\

c) 
$\displaystyle{\int^{100}_0 \frac{dx}{x^2+1} \, }$
\\ 

%d) $\displaystyle{\int \tan x \ln |\cos x| \, dx=}$
\vspace{3.5in}

{\bf 7 [12P])} Evaluate the sum $ 
\displaystyle{\sum_{j=1}^{20}\left( 2j  +\, 5\right) =
\hbox  to 2cm{\hrulefill}}$

\vspace{4.25in}

{\bf 8 [20P])} Sketch the graph of the function
${\displaystyle f(x) = \frac{x^2}{x-1}}$. Determine the domain and
range. If there are any, then identify and label all
extrema ([4P]), inflection points ([4P]), intercepts
([4P]), and asymptotes ([4P]). Indicate the concave structure
clearly ([4P]).

\vspace{4.25in}


{\bf 9 [12P])}
Use linear approximation to estimate
$\displaystyle{8.001^{1/3}\approx\hbox  to 3cm{\hrulefill} }$
%\newpage

\vspace{3.25in}


{\bf 10 [15P])} Find the length of the arc
$\displaystyle{y = 0.25 + 6t^{3/2}}$
from $\displaystyle{t=0}$ to $\displaystyle{t = 2}$.
\\

The length is: \hbox  to 3cm{\hrulefill}

\vspace{3.25in}




{\bf 11 [20P])} Sketch the region bounded by the curves
$\displaystyle{y=x^3+x^2-6x}$ and ${\displaystyle y = 0}$ and find its 
total area.
\\

The area is : \hbox  to 3cm{\hrulefill}


\vspace{3.25in}

{\bf 12 [15P])} Express as an  integral the volume of the solid that is generated
by rotating the plane region bounded
by the curves  $\displaystyle{y= x^2}$ and
${\displaystyle x =  y^2}$  around the $y$-axes.
Do {\em not} evaluate the integral. 
\\

%The volume is: \hbox  to 3cm{\hrulefill} 

\vspace{3.25in}

%{\bf 14 [15P])} Find the volume of the solid that is generated
%by rotating the plane region bounded
%by the curves  $\displaystyle{y= x -x ^2,\quad
%y = 0}$ around the axes  $y=-1$.

\end{document}




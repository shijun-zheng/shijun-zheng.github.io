\documentclass[12pt]{article}
\usepackage{amsmath,amssymb, amsfonts}
\usepackage{latexsym}
\pagestyle{empty}

\renewcommand{\baselinestretch}{1.2}
\parindent0em

\begin{document}

\begin{large}
\begin{bf}
\hspace{.75in}
{MATH 1550-24, Test 3} %. Fri. Apr.30, 2004}
\hspace{1in}
\parbox{1in}{ Name \newline Id}
\end{bf}
\end{large}

\vspace{0.2in}
%\noindent
%Use one page for each problem or sub-problem of the nine numbered questions(use %back of the page if necessary).

Put the question number (including its sub-problem number, if any) for each 
problem on your answer sheet. 
Put a box around the final answer to a question.

\vspace*{.02in}
For full credit you must \emph{show your work}. You must have enough
written work, including explanations when called for, to justify your
answers. Incomplete solutions may receive partial credit if you have
written down a reasonable partial solution. %Correct answers may not
                                            %receive credit if
                 

\vspace{.25in}

{\bf 1 [20P])} Evaluate the indefinite integrals
\begin{enumerate}
\item \; ${\displaystyle \int (x+3)^{5/2}dx }$.
\item \; ${\displaystyle \int \frac{x}{4 - x^2}\, dx \,  }$
\item \; ${\displaystyle \int \frac{1 + e^{2x}}{e^{x}}\, dx \, }$
%\item )\; ${\displaystyle \int \frac{1 + \ln x}{x}\, dx}$.
\item \; ${\displaystyle \int  \csc x\cot x dx} $
\end{enumerate}


\newpage

{\bf 2.} Assume that ${\displaystyle [x_{i-1},x_i]}$ denotes the
$i^{th}$ subinterval of a subdivision of $[0,2]$,
which is divided into $n$ subintervals having the same length ${\displaystyle
\Delta x = \frac{2}{n}}$. \\
{\bf a [5P])} Evaluate the Riemann sum ${\displaystyle
\sum_{i=1}^n \left(x_i -0.5\right)\Delta x}$.

{\bf b [5P])} What is ${\displaystyle
\lim_{n\to \infty}
\sum_{i=1}^n \left(x_i -0.5\right)\Delta x}$.

{\bf c [5P])} Give an integral 
(but do not evaluate it) that is approximated by the Riemann sum in {\bf a}).

\vspace{.5in}

%\item ${\displaystyle \int x^2 \cos (x^3)\, dx }$. 
%\item ${\displaystyle \int \frac{({\rm ln}(x))^2}{x}\, dx \, = }$

\newpage

{\bf 3 [20P])} Evaluate the definite integrals
\begin{enumerate}
\item \;${\displaystyle \int_e^{100} \frac{dx}{ x\ln x} }$.
%\item \;${\displaystyle \int_{-10}^{10} \frac{x}{x^2 + 1} \, dx }$
\item \;${\displaystyle \int^{1/2}_{-1/2} \frac{dt}{\sqrt{1-t^2}} }$.
%\item )\;${\displaystyle \int_0^{\pi / 2} \sin x \cos x\, dx}$.
%\item )\;${\displaystyle \int_{-1}^1 \frac{x}{x^2 + 2} \, dx }$.
\end{enumerate}

\vspace{4.25in}

{\bf 4 [15P])} Find the derivative of the function
${\displaystyle F(x) = \int_0^{x^3}  \cos (t^2) dt }$.

\vspace{2.67in}

\newpage

{\bf 5 [15P])} Find the total area of the bounded region given by the
${\displaystyle x-}$axis and the graph of the function
${\displaystyle f(x) = x^3-x }$ on $[-1,1]$.

\vspace{3.90in}
 
%{\bf 7 [12P])} Find the volume of the solid that is generated by the plane region bounded by
%${\displaystyle y = 9 - x^2 }$ and ${\displaystyle y = x^2 + 1}$ rotated
%around the ${\displaystyle x-}$axis.

%\vspace{.35in}

{\bf 6 [15P])} Use either the method of disk/washer or the method of 
cylindrical shells to express as an integral
the volume of the solid generated by revolving around the
$y$-axis the region bounded by $y= x^2$ and
$y=\sqrt{x}$.  Do {\em not} evaluate the integral.


%{\bf 9 [15])} Find
%the lenght of the smooth arc  ${\displaystyle y = \frac{1}{x} x^3 + \frac{1}{2x}}$ from $x=1$ to $x=3$.

%{\bf 3 [12P])} Solve the initial value problem
%${\displaystyle \frac{dy}{dx} = \frac{1}{2yx},\quad
%y(1) =2}$.
%\\ %\nlc{3.5}

%{\bf 9 [11P])} Find the work done be the force ${\displaystyle
%F(x) = \frac{10}{x^2}}$
%in moving a particle along the $x$-axis from $x=1$ to $x=10$.

%\vspace{.35in}

%{\bf 9 [8P])} Find the work done by the force $
%F(t) = \frac{10}{2t^2+1}$
%in moving a particle along the $x$-axis from $t=0$ to $t=1.5$

%{\bf 10 [15P])} Use the method of cylindrical shells to find
%the volume of the solid generated by revolving around the
%$y$-axis the region bounded by $y= x^2$ and
%$y = 8 - x^2$.


\end{document}
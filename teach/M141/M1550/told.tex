\documentclass[12pt]{article}
%\pagestyle{headings}
\pagestyle{empty}
%\input ../danp
%\catcode`\"=\active
\textwidth16cm
\textheight23cm
\oddsidemargin0cm
\evensidemargin-4.5mm
\topmargin-10mm
%\input ../../makromat
\renewcommand{\baselinestretch}{1.2}
\parindent0em
\begin{document}
\centerline{\large\bf NAME :\hbox to 12cm{\hrulefill}}
%\nl{5}

\centerline{\bf MATH 1550-10, Test \# 2, Fr. March 10 1995}
\centerline{\bf Show your work}
%\nl{5}
{\bf 1 [16P])}  Find the derivative of the following functions:
%\nl{5}
a) $\displaystyle{f(x) = \cos (x)+\frac{2}{x} \quad f^\prime(x) =}$
%\hbox to 5cm{\dotfill}
\fbox{\rule[-7mm]{0cm}{1.5cm}\hbox to 8cm{}}
\\ %\nlc{2.5}
b) $\displaystyle{f(x) = \left(\frac{1}{x^2+1}\right)^2\quad f^\prime(x) =}$
\fbox{\rule[-7mm]{0cm}{1.5cm}\hbox to 8cm{}}
\\ %\nlc{3}
{\bf 2) [16P])} Find $dy/dx$ for the following functions:
\\ %\nl{5}
a) $ \displaystyle{y = [1  + \ln (x)]^3. \quad dy/dx =}$\fbox{\rule[-7mm]{0cm}{1.5cm}\hbox to 8cm{}}
\\ %\nlc{2.5}
b) $\displaystyle{y = e^x(x^2+2)\quad dy/dx =}$
\fbox{\rule[-7mm]{0cm}{1.5cm}\hbox to 8cm{}}
%\newpage
{\bf 3 [14P])} 
Find the slope of the tangent line to the graph
of the equation 
$\displaystyle{xy^3+x^2y=10}$
at the point $(1,2)$.
The slope is: \fbox{\rule[-7mm]{0cm}{1.5cm}\hbox to 8cm{}}
\\ %\nlc{8}
%%\newpage
{\bf 4 [14P])} Use the Newton's method to
find the first 3 digits in $\displaystyle{\sqrt{7}}$.
{\bf Show each step in your calculations!}
\\
\fbox{\rule[-7mm]{0cm}{1.5cm}$\displaystyle{
x_0 = }$\hbox to 3cm{}}\hfill
\fbox{\rule[-7mm]{0cm}{1.5cm}$\displaystyle{
\sqrt{7}\approx }$\hbox to 4cm{}}
%\newpage
{\bf 5 [16P])} Use linear approxination to estimate the 
following
numbers: 
\\ %\nl{5}
a) $\displaystyle{e^{0.2} \approx }$
\fbox{\rule[-7mm]{0cm}{1.5cm}\hbox to 6cm{}}
\\ %\nlc{5}
b) $9^{1/3}\displaystyle{\approx}$
\fbox{\rule[-7mm]{0cm}{1.5cm}\hbox to 6cm{}}
\\ %\nlc{5.5}
{\bf 7 [8P])}  Find the linear aproximation to the
function $\displaystyle{f(x) = (1+x)^{5/3}}$ near the point $a=0$.
%\\ %\nlc{1.5}
%\newpage
{\bf 7 [16P])}  Let $\displaystyle{f(x) = x^4-2x^2+1}$.\\
a) Find the intervals on the $x$-axis on which the funtion is 
increasing as
well as those on which it is decreasing.
\\ %\nlc{4}
b) Find the absolut maximum and absolut minimum on the interval
$[-2,2]$.
\\ %\nlc{4}
{\bf Bonus problem, 9P}: Determine for the following 3 functions the
open intervals on the $x$-axis on which each function
is increasing and those where it is decreasing. Then use
this information to match the function to
is graph:
\\ %\nl{5}
A) $\displaystyle{f(x) = xe^x }$,\hfill B) $\displaystyle{f(x) =
\frac{1}{3}x^3+x^2+12}$
\hfill
C) $\displaystyle{f(x) = x^4-\frac{8}{3}x^3-2x^2+8x}$.
%\newpage
%\newpage
MATH 1550-10, Test \# 3, T. April 17 1995
\hfill Name:\hbox to 6cm{\hrulefill}
\\ %\nl{5}
{\bf 1 [20P]) (Antiderivatives)}  Find the following 
antiderivatives:
\\ %\nl{5}
a) $\displaystyle{\int \frac{2x^4 - 3x + 5}{x^5}\, dx 
=}$
\\ %\nlc{1.5}
b) $\displaystyle{\int x e^{9x^2}\, dx=}$
\\ %\nlc{1.5}
c) $\displaystyle{\int x(2x + 1)^5 =}$
\\ %\nlc{1.5}
d) $\displaystyle{\int x^5 + \frac{2}{\sqrt{x}}\, dx = }$
\\ %\nlc{2}
{\bf 2 [18P])} Evaluate the following integrals
\\ %\nl{5}
a) $ \displaystyle{\int_1^2\frac{1 + \ln x}{x}dx=}$
\\
 %\nlc{2}
b) $\displaystyle{\int_{0}^{p /6} \sin (3x) \cos^3 (3x)dx =}$
\\ %\nlc{2}
%%\newpage
{\bf 3 [12P])} Solve the initial value problem
$\displaystyle{\frac{dy}{dx} = 3 \sin (x) + 2x,\quad y(0) = 4}$.
%\\ %\nlc{3.5}
%\newpage
{\bf 4 [10P])} Compute the sum
$\displaystyle{\sum_{i=1}^{99}\left(\frac{1}{199}\, i^2 -i - 
3\right)}=$
\\ %\nlc{2.5}
{\bf 5 [15P])} Sydney drops a rock into a well in
which the water surface is $400$ ft below the ground. How long
does it take the rock to reach the water surface? How fast is the
rock moving as it penetrates the water surface? 
%\\ %\nlc{1.5}
\\ %\nl{5}
{\bf Solution:} The time is =  \dotfill \hfill The speed is = 
\dotfill 
\\ %\nlc{3}
{\bf 6 [10P])} Evaluate the integral
$\displaystyle{\int_{-2}^{3} |x^3-1|\, dx = }$.
\\ %\nlc{3.5}
{\bf 8 [15P])}  Find the area of the region bounded by the
graphs of $y=x^2$ and $y=8-x^2$.
\\ %\nl{5}
{\bf Solution:} The area is: 
%\newpage
MATH 1550-10, Take Home Test, due Monday\\
%\hfill Name:\hbox to 6cm{\hrulefill}\hfill
\\ %\nl{5}
{\bf 1)} Write the equation for the tangent line and the
normal line to the graph of $\displaystyle{y = x\sin x + 1}$
at the point $P(0,1)$.
\\ %\nlc{2}
{\bf 2)} Find the first two derivatives of the functions:
\\ %\nl{5}
a) $ \displaystyle{f(x) = \frac{\cos (x)}{x} + x^6}$
\\ %\nl{5}
b) $\displaystyle{f(x) = xe^{x^2}}$
\\ %\nlc{2}
%%\newpage
{\bf 3)}  Find the limit $\displaystyle{
\lim_{x\to 4}\frac{x^2-4}{x(2-x)}}$.
\\ %\nlc{2}
{\bf 4)} A farmer has $600 m$ of fencing with which to
enclose a rectangular pen. What is the
maximum area that can be enclosed?
\\ %\nlc{2}
{\bf 5)} a) Use linear approximation to extimate
$\displaystyle{\frac{1}{15^{1/4}}}$.
\\ %\nl{5}
b) Use Newton's method to find the solution
to the equation $\displaystyle{x^3-x-7= 0}$ in
the interval $\displaystyle{[3,4]}$.
%\newpage
{\bf 6)} Sketch the graph of the function 
$\displaystyle{\frac{e^x}{x-2}}$. Identify and lable all
extrema, inflection points, intercepts and asymptotics.
\\ %\nlc{6}
{\bf 7)} Evaluate the following antiderivatives:
\\ %\nl{5}
a) 
$\displaystyle{\int x^4 + \cos x - \frac{1}{x^{1/3}}\, dx}$.
\\ %\nl{5}
b) $\displaystyle{\int 2x^2\sin (x^3)\, dx}$.
\\ %\nl{5}
c) $\displaystyle{\int \frac{\ln (1/x)}{x}\, dx}$.
\\ %\nlc{2}
{\bf 8)} Evaluate the following integrals:
\\ %\nl{5}
a)  $\displaystyle{\int_0^2 x^2 - 1\, dx}$.
\\ %\nl{5}
b)  $\displaystyle{\int_0^{p /2} {\rm sec}^2(4x)\, dx}$.
%\newpage
{\bf 9)} Sketch the region bounded by the curves
$\displaystyle{y=x,\quad y= x^3,}$
and find its area.
\\ %\nlc{5}
{\bf 10)} Find the volume of the solid that is generated
by rotating the plane region bounded
by the curves  $\displaystyle{y= x^2,\quad
y = 8 - x^2}$ around the line $y=-1$.
\\ %\nlc{2}
{\bf 11)} Lynda shoots an arrow straight upwards from the
roof of a 400 ft house with initial velocity $320$ ft/s.
\\ %\nl{5}
a) How high is the arrow after $3$ s?
\\
b) At what time is the arrow exactly $1600$ ft above
the ground?
\\
c) How many seconds after its release does the arrow strike the
ground and what is it speed?
\\ %\nlc{2}
{\bf 12)} A town had a population of 10,000 in 1970 and
a population of 12,000 in 1990. Assume that its population
will continue to grow exponentially at a constant
rate. What is the expected population in the year 2010?
\\ %\nlc{2}
{\bf 13)} Find the derivative of the functions:
\\ %\nl{5}
a) $\displaystyle{f(x) = \log_{10} \cos (x)}$.
\\
b) $\displaystyle{f(x)= 2^x3^{x^2}}$.
%\newpage
MATH 1550-10, Final, Wedensday  May 10
\\ %\nlc{1}
{\bf Name:}\hbox to 10cm{\hrulefill}
\\ %\nl{2}
\centerline{\Large\bf * * *}
\\ %\nl{3}
{\bf 1 [15P])} Write the equation for the tangent line and the
normal line to the graph of $\displaystyle{y = x \cos (x)}$
at the point $P(\frac{p}{2},0)$.
\\ %\nl{5}
Tangent line:\hbox to 3.5cm{\hrulefill}\hspace{1.5cm}
Normal line:\hbox to 3.5cm{\hrulefill}\hfill
\\ %\nlc{5}
%%\newpage
{\bf 2 [15P])}  Find the limit $\displaystyle{
\lim_{x\to 2}\frac{x^3 - 7x + 6}{x^2-x - 2}=\hrulefill}$
\\ %\nlc{2}
{\bf 3 [15P])} Find the maximum and minumum values of
the function 
$\displaystyle{f(x) = |x^2-1| +\frac{1}{2}x^2}$ on the
closed interval $\displaystyle{[0,2]}$.
\\ %\nl{5}
The maximum is:\hbox to 2cm{\hrulefill}\hspace{1.5cm}
The minimum value is: \hbox to 2cm{\hrulefill}
%\newpage
{\bf 4 [20P])} Find the first two derivatives of the function
$\displaystyle{f(x) = \frac{1}{x} + xe^{x^2} + 5^x}$.
\\ %\nl{5}
$\displaystyle{f^\prime(x) = }$
\\ %\nl{5}
$\displaystyle{f^{\prime\prime}(x) = }$
\\ %\nlc{4}
{\bf 5 [15P])} Evaluate the sum $ 
\displaystyle{\sum_{j=1}^{15}\left(
\frac{3}{20} j\,  -\, 2\right) = }$
\\ %\nlc{4}  
{\bf 6 [24P])}  Evaluate the following antiderivatives:
\\ %\nl{5}
a) 
$\displaystyle{\int \left( x^5 + x^{3/2} - \frac{1}{x^4}
\right)\, dx =}$
\\ %\nl{5}
b) $\displaystyle{\int \tan x \ln |\cos x| \, dx=}$
%\\ %\nl{5}
%c) $\displaystyle{\int \frac{x}{2x^2 + 3}\, dx}$.
%\\ %\nlc{6}
%\newpage
{\bf 7 [26P])} Evaluate the following integrals:
\\ %\nl{5}
a)  $\displaystyle{\int_{-1}^0 x(1\, + \, x)^5\, dx = }$
\\ %\nl{5}
%b)  $\displaystyle{\int_4^9\frac{( 1 + \sqrt{x})^3}{\sqrt{x}}\, 
%dx=}$
%\\ %\nl{5}
b)  $\displaystyle{\int_1^2\frac{ x^3 - 2x^2 + x}{x}\, 
dx=}$
\\ %\nlc{7}
{\bf 8 [15P])} Solve the initial value problem
$\displaystyle{ \frac{dy}{dx} = x^2 + \cos (x),\quad y(0) = 3}$.
\\ %\nl{5}
$\displaystyle{y(x) = }$
\\ %\nlc{4}
{\bf 9 [10P])}
Use linear approximation to estimate
$\displaystyle{\frac{1}{28^{2/3}}= }$
%\newpage
{\bf 10 [15P])} Sketch the region bounded by the curves
$\displaystyle{y=x^2,\quad y=  2x}$ and find its area.
\\ %\nl{5}
The area is :
\\ %\nlc{6}
%%\newpage
{\bf 11 [15P])} Find the volume of the solid that is generated
by rotating the plane region bounded
by the curves  $\displaystyle{y= x -x ^2,\quad
y = 0}$ around the axes  $y=-1$.
\\ %\nlc{6}
{\bf 12 [15P])} Find the lengt of the arc
$\displaystyle{y = \frac{2}{3}\left(1 + x^2\right)^{3/2}}$
from $\displaystyle{x=0}$ to $\displaystyle{x = 2}$.
\\ %\nl{5}
The length is : 
\\ %\nlc{2}
{\bf 11)} Lynda shoots an arrow straight upwards from the
ground with initial velocity $320$ ft/s.
\\ %\nl{5}
a) How high is the arrow after $3$ s?
\\
b) At what time is the arrow exactly $1200$ ft above
the ground?
\\
c) How many seconds after its release does the arrow strike the
ground and what is it speed?
\\ %\nlc{2}
{\bf 12)} A town had a population of 25,000 in 1980 and
a population of 30,000 in 1990. Assume that its population
will continue to grow exponentially at a constant
rate. What is the expected population in the year 2000?
\\ %\nlc{2}
{\bf 13)} Find the derivative of the functions:
\\ %\nl{5}
a) $\displaystyle{f(x) = \log_{10} \cos (x)}$.
\\
b) $\displaystyle{f(x)= 2^x3^{x^2}}$.
%\end{document}
%\newpage
\centerline{\large\bf NAME :\hbox to 12cm{\hrulefill}}
\\ %\nl{5}
\centerline{\bf MATH 1550-15, Test \# 1, Fr. Sept. 15. 1995}
\centerline{\bf Show your work}
\\ %\nl{5}
{\bf 1 [11P])}  Write the equation for the tangent line and the normal line
to curves
$\displaystyle{f(x) = x^2+3x + 1}$ at the
point $\displaystyle{P(1,5)}$:
\\ %\nlc{1}
{\bf Tangent line: }
\\ %\nl{5}
{\bf Normal line: }
\\ %\nl{5}                                  
{\bf 2) [32P])}  Find the following limits or determine if they do not exists:
\\
a) $\displaystyle{\lim_{x\to 0}x\sin \left(\frac{1}{x^2}\right)=}$
\\ %\nl{5}
b) $\displaystyle{\lim_{x\to 0}\frac{\sin 3x}{5x}=}$
\\ %\nl{5}
c) $\displaystyle{\lim_{x\to 1}\frac{x^2-1}{x^2+x-2}=}$
\\ %\nl{5}
d) $\displaystyle{\lim_{x\to 2}\frac{x^2+2x-3}{x^2+x-6}=}$
\\ %\nlc{1.5}
%%\newpage
{\bf 3 [6P])}  Expain if the equation $\displaystyle{x^3+x + 1}$ has a solution in the
interval $[-1,0]$ or not.
\\ %\nlc{3}
{\bf 4 [11P])} Explain why the function
$\displaystyle{f(x) = \frac{\sin (x^2)}{x^2}}$ if $x\not= 0$ and $f(0) = 0$ is continuous or
not:
%\newpage
{\bf 5 [11P])}  Let $\displaystyle{f(x) = \frac{1-x}{1-x^2}}$. Where
is $f(x)$ defined?  Find the left hand side and the right hand side limits at the
points where $f(x)$ is not defined. Is it possible to assign a value to $f(x)$ at
those points such that $f(x)$ is continuous at the point?
\\ %\nlc{3.5}
{\bf 6 [18P])} Find the derivative of the following functions:
\\ %\nl{5}
a) $\displaystyle{f(x) = x\cos (x) + x^2 + \frac{1}{\sin x}}$. $\displaystyle{f^\prime(x) = }$
\\ %\nlc{1.5}
b) $\displaystyle{f(x) = \frac{x^2 + 3x}{x^2+1}}$.
$\displaystyle{f^\prime(x) = }$
\\ %\nlc{1.5}
c) $\displaystyle{f(x) = x\cos (x) \sin (x)}$. $\displaystyle{f^\prime(x) = }$
\\ %\nlc{1.5}
{\bf 7 [11P])} A water bucket containing $10$ gal of water
develops a leak at time
$t=0$. The volume $V$ of water
in the bucket $t$ seconds later is given by
\[ V(t) = 10\left(1 - \frac{t}{100}\right)^2\]
until the bucket is empty at time $t=100$.
%\npar
a) At what rate is water leaking from the bucket after
exactly $1$ min.
%\npar
b) What is the average rate of change of $V$ from time $t=0$
and $t=50$ and from time $t=0$ and $t=100$?
%\newpage
\centerline{\large\bf NAME :\hbox to 12cm{\hrulefill}}
\\ %\nl{5}
\centerline{\bf MATH 1550-15, Test \# 2, Wed. Oct. 18. 1995}
\centerline{\bf Show your work. Circle your solution}
\\ %\nl{5}
{\bf 1 [12P])} Find the derivative of the following two functions:
\\ %\nl{5}
a) ${\displaystyle f(x) = x^2\cos (x^2 + 1)}$.
${\displaystyle f^\prime (x) =  }$
\\ %\nlc{2}
b) ${\displaystyle f(s) = (e^s + 1)^3 }$. ${\displaystyle f^\prime(s) = }$
\\ %\nlc{2}      
{\bf 2) [10P])} Find the slope $m$ of the tangent line to the graph
of the equation 
$\displaystyle{xy^3+x^2y=10}$
at the point $(1,2)$.
The slope is: $m = $
\\ %\nlc{3}
{\bf 3 [8P])} Use linear approximation to find $63^{1/3}$.
\\ %\nlc{3}
{\bf 4 [8P])} Find the linear approximation to the function
${\displaystyle f(x) = \frac{1}{(1 + x)^{2/3}}}$
near the point $a = 0$.
%\newpage
{\bf 5 [15P])} Find the intervals where the function
${\displaystyle f(x) = x^4 - 2x^2 +1}$ is increasing as well as
those on which it is decreasing.
\\ %\nlc{3}
{\bf 6 [12P])} Find the absolut maximum and absolut minimum of
the function ${\displaystyle f(x) = x^2 - 4x + 1}$ on the
interval $[-2,2]$.
\\ %\nlc{3}
{\bf 7 [10P])} Calculate the first two derivatives of the function
$f(x) = x \ln x$.

\hfill${\displaystyle f^\prime(x) = }$\hspace{5cm} ${\displaystyle f^{\prime\prime}(x) =
}$\hspace{5cm}\hfill
\\ %\nlc{3}
{\bf 8 [25P])} Sketch the graph of the function
${\displaystyle f(x) = \frac{x}{x+1}}$. Identify and label all
extrema ([4P]), inflection points ([4P]), intercepts
(4 Points), and asymptotes ([4P]). Indicate the concave structure
clearly ([4P]).
%\newpage
\centerline{\large\bf NAME :\hbox to 12cm{\hrulefill}}
\\ %\nl{5}
\centerline{\bf MATH 1550-15, Test \# 3, Th. Nov. 9. 1995}
\centerline{\bf Show your work. Circle your solution}
\\ %\nl{5}
{\bf 1 [10P])} Evaluate the sum ${\displaystyle \sum_{k=1}^{10}\left(4 k - 3\right)}$.  
\\ %\nlc{1.5}    
{\bf 2)} Assume that ${\displaystyle [x_{i-1},x_i]}$ denotes the
ith subinterval of a subdivision of ${\displaystyle [0,1]}$
into $n$ subintervals all which the same length ${\displaystyle
D (x) = \frac{1}{n}}$.
%\npar
{\bf a [7P])} Evaluate the Riemann sum ${\displaystyle
\sum_{i=1}^n \left(x_i^2 + 4\right)D (x)}$.
\\ %\nlc{2}
%\npar
{\bf b [8P])} What is ${\displaystyle
\lim_{n\to \infty}
\sum_{i=1}^n \left(x_i^2 + 4\right)D (x)}$.
\\ %\nlc{2}
{\bf 3 [18P])} Find the antiderivatives
\begin{enumerate}
\item ${\displaystyle \int \frac{6}{x^{3/2}}dx}$.
\item ${\displaystyle \int \frac{1 + \ln x}{x}\, dx}$.
\item ${\displaystyle \int x \cos (x^2)\, dx }$.
\end{enumerate}
 %\nlc{2}
%\newpage

{\bf 4 [18P])} Evaluate the integrals
\begin{enumerate}
\item ${\displaystyle \int_1^3 \left( x - 1\right)^3 dx}$.
\item ${\displaystyle \int_0^{p / 2} \sin x \cos x\, dx}$.
\item ${\displaystyle \int_0^1 \frac{x}{x^2 + 2} \, dx }$.
\end{enumerate}
 %\nlc{2}

{\bf 5 [9P])} Find the derivative of the function
${\displaystyle F(x) = \int_0^{e^x} \frac{ ( \ln t)^2}{t}\, dt}$.
\\ %\nlc{2}
{\bf 6 [15P])} Find the total area of the bounded region given by the
${\displaystyle x-}$axis and the graph of the function
${\displaystyle f(x) = x^3 + x^2 - 6x }$.
\\ %\nlc{4} 
{\bf 7 [15P])} Find the volume of the solid that is generated by the plane region bounded by
${\displaystyle y = 9 - x^2 }$ and ${\displaystyle y = x^2 + 1}$ rotated
around the ${\displaystyle x-}$axis.
%\newpage
\centerline{\large\bf NAME :\hbox to 12cm{\hrulefill}}
\\ %\nl{5}
\centerline{\bf MATH 1550-15, Test \# 4, fall 1995. Make up}
\centerline{\bf Show your work. Circle your solution}
\\ %\nl{5}
{\bf 1 [15P])} Use the method of cylindrical shells to find
the volume of the solid generated by revolving around the
$y$-axis the region bounded by $y= 25 - x^2$ and
$y = 0$.
\\ %\nlc{4}    
{\bf 2 [15])} Find
the lenght of the smooth arc  ${\displaystyle y = \frac{1}{x} x^3 + \frac{1}{2x}}$ from $x=1$ to $x=3$.
\\ %\nlc{3}
{\bf 3 [12P])} Solve the initial value problem
${\displaystyle \frac{dy}{dx} = \frac{1}{2yx},\quad
y(1) =2}$.
\\ %\nlc{3.5}
{\bf 4 [11P])} Find the work done be the force ${\displaystyle
F(x) = \frac{10}{x^2}}$
in moving a particle along the $x$-axis from $x=1$ to $x=10$.
%\newpage
{\bf 5 [15P])} Evaluate the limits
\\ %\nl{5}
a) ${\displaystyle \lim_{t\to \infty} x^2e^{-x}=}$.
\\ %\nlc{2}
b) ${\displaystyle \lim_{t\to \infty} \frac{({\rm ln}(x))^2}{x} = }$.
\\ %\nlc{2}
{\bf 6 [16P])}
Evaluate the following integrals
\\ %\nl{5}
a) ${\displaystyle \int \frac{({\rm ln}(x))^2}{x}\, dx \, = }$
\\ %\nlc{1.5}
b) ${\displaystyle \int \frac{x + e^{2x}}{x^2+e^{2x}}\, dx \, = }$
\\ %\nlc{2}
{\bf 7 [16P])} Differentiate the following functions:
\\ %\nl{5}
a) ${\displaystyle f(x) = 2^{\sqrt{x}}\, ,\quad f^\prime (x) =  }$
\\ %\nlc{1.5}
b) ${\displaystyle g(x)  = \frac{\sqrt{x^2+1}}{(1+x)^{3/2}}\, , \quad g^\prime (x) =}$
%\newpage
\centerline{\large\bf NAME :\hbox to 12cm{\hrulefill}}
\\ %\nl{5}
\centerline{\bf MATH 1550-15, Test \# 4, Mo. Dec. 4. 1995}
\centerline{\bf Show your work. Circle your solution}
\\ %\nl{5}
{\bf 1 [15P])} Use the method of cylindrical shells to find
the volume of the solid generated by revolving around the
$y$-axis the region bounded by $y= x^2$ and
$y = 8 - x^2$.
\\ %\nlc{4}    
{\bf 2 [15])} Find
the lenght of the smooth arc  ${\displaystyle y = \frac{2}{3}\left(1 + x^2\right)^{3/2}}$ from $x=0$ to $x=1$.
\\ %\nlc{3}
{\bf 3 [12P])} Solve the initial value problem
${\displaystyle \frac{dy}{dx} = \frac{1}{2y},\quad
y(0) =1}$.
\\ %\nlc{3.5}
{\bf 4 [11P])} Find the work done be the force ${\displaystyle
F(x) = \frac{1}{(2x+1)^2}}$
in moving a particle along the $x$-axis from $x=0$ to $x=1$.
%\newpage
{\bf 5 [15P])} Suppose that the fish population $P(t)$ in a
lake is given by the differential equation
${\displaystyle \frac{dP}{dt} = - k\sqrt{P(t)}}$. If
there were $400$ fishes in the lake at $t=0$ and
$10$ weeks later only
$100$, how many are there after $15$ weeks and how long will it
take all the fish in the lake to die.
\\ %\nlc{6}
{\bf 6 [16P])}
Evaluate the following integrals
\\ %\nl{5}
a) ${\displaystyle \int \frac{x}{1 + 3x^2}\, dx \, = }$
\\ %\nlc{1.5}
b) ${\displaystyle \int \left( xe^{1-x^2} + 4^x\right)\, dx \, = }$
\\ %\nlc{2}
{\bf 7 [16P])} Differentiate the following functions:
\\ %\nl{5}
a) ${\displaystyle f(x) = 3^x + {\rm log}_{10}(x)\, ,\quad f^\prime (x) =  }$
\\ %\nlc{1.5}
b) ${\displaystyle g(x)  = x e^{\cos x}\, , \quad g^\prime (x) =}$
\end{document}
 



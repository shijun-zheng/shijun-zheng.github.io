\documentclass[12pt]{article}
%\pagestyle{headings}
\pagestyle{empty}
%\input ../danp
%\catcode`\"=\active
\textwidth16cm
\textheight23cm
\oddsidemargin0cm
\evensidemargin-4.5mm
\topmargin-10mm
%\input ../../makromat
\renewcommand{\baselinestretch}{1.2}
\parindent0em
\begin{document}
\begin{center}
{\large\bf MATH 1550-15, Final, Th, Dec. 14, 3:00-5:00 PM}
%\nl{5}
{\large\bf NAME :\hbox to 12cm{\hrulefill}}
%\nl{4}
{\bf Do you want me to post your grade with SS\# : Yes \hbox to 2cm{\hrulefill}
No \hbox to 2cm{\hrulefill}
%\nl{5}
Show your work, circle your solutions
}
\end{center}
%\nl{5}
{\bf 1 [12P])} Write the equation for the tangent line and the
normal line to the graph of ${\displaystyle f(x) = x^2 + 4 x +2 
}$
at the Point $P(1,7)$. 
%\nl{5}
Tangent line:\hbox to 3.5cm{\hrulefill}\hfill
Normal line:\hbox to 3.5cm{\hrulefill}\hfill
%\nlc{3}
{\bf 2 [12P])}  Find the limits
%\nl{5}
a) $\displaystyle{
\lim_{x\to 2}\frac{x^3 - 7x + 6}{x^2-x - 2}=
\hbox  to 3cm{\hrulefill}}$
%\nlc{2}
b) ${\displaystyle \lim_{x\to 0}\frac{\sin 3x}{5x}=
\hbox to 3cm{\hrulefill}}$
%\nlc{2}
{\bf 3 [14P])} Find the maximum and minumum values of
the function 
$\displaystyle{f(x) = x + \frac{4}{x}}$ on the
closed interval $\displaystyle{[1,3]}$.
%\nl{5}
The maximum is:\hbox to 3cm{\hrulefill}\hspace{1.5cm}\hfill
The minimum value is: \hbox to 3cm{\hrulefill}
\newpage
{\bf 4 [18P])} Find the derivative of the following functions
%\nl{5}
a) ${\displaystyle f(x) = \frac{1}{x} + xe^{x^2}. \quad f^\prime (x) = }$
%\nlc{2}
b) ${\displaystyle f(x) =  3^x + {\rm log}_{10}(x). \quad
f^\prime (x) = }$
%\nlc{2}
{\bf 5 [18P])}  Evaluate the following antiderivatives:
\\ %\nl{5}
a) 
$\displaystyle{\int \left( 4^x + x^{3/2} - \frac{1}{x^4}
\right)\, dx =}$
%\nlc{2} %\nl{5}
b) $\displaystyle{\int \frac{1 + \ln (x)}{x} \, dx=}$
%\nlc{2}
{\bf 6 [20P])} Evaluate the following integrals:
\\ %\nl{5}
a)  $\displaystyle{\int_{0}^1 (1 - x)^5\, dx =}$
%\nlc{3}
b) ${ \int_0^{p /2} (\sin x)^2 \cos x\, dx\, = }$
\newpage
{\bf 7 [12P])} Evaluate the sum $ 
\displaystyle{\sum_{j=1}^{10}\left( 2 j\,  -\, 1\right) =
\hbox  to 2cm{\hrulefill}}$
%\nlc{3} 
{\bf 8 [24P])} Sketch the graph of the function
${\displaystyle f(x) = \frac{x}{x+1}}$. If there are any, then identify and label all
extrema ([4P]), inflection points ([4P]), intercepts
([4P]), and asymptotes ([4P]). Indicate the concave structure
clearly ([4P]).
%\nlc{7}
{\bf 9 [12P])}
Use linear approximation to estimate
$\displaystyle{28^{2/3}\approx\hbox  to 3cm{\hrulefill} }$
%\newpage
%\nlc{2}
{\bf 10 [15P])} Sketch the region bounded by the curves
$\displaystyle{y=x^2}$ and ${\displaystyle y =  2x}$ and find its area.
%\nl{4}
The area is : \hbox  to 3cm{\hrulefill}
\newpage
{\bf 11 [15P])} Find the volume of the solid that is generated
by rotating the plane region bounded
by the curves  $\displaystyle{y= x}$ and
${\displaystyle y =  x^2}$  around the $y$-axes.
%\nl{4}
The volume is: \hbox  to 3cm{\hrulefill} 
%\nlc{6}
{\bf 12 [15P])} Find the length of the arc
$\displaystyle{y = \frac{2}{3}\left(1 + x^2\right)^{3/2}}$
from $\displaystyle{x=0}$ to $\displaystyle{x = 1}$.
%\nl{4}
The length is: \hbox  to 3cm{\hrulefill}
%\nlc{4}
{\bf 13 [14P])} Suppose that the fish population $P(t)$ in a
lake is given by the differential equation
${\displaystyle \frac{dP}{dt} = - k\sqrt{P(t)}}$. If
there were $576$ fish in the lake at $t=0$ and
$4$ weeks later only
$144$, how many are there after $5$ weeks and how long will it
take all the fish in the lake to die?
\end{document}


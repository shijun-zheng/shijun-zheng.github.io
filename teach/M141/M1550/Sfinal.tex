\documentclass[12pt]{article}
\usepackage{amsmath,amssymb, amsfonts}
\usepackage{latexsym}
\pagestyle{empty}

\renewcommand{\baselinestretch}{1.2}
\parindent0em

\begin{document}
%\begin{center}
\centerline{\bf MATH 1550-24, Sample Test}%. Thur. May 13, 2004}
\centerline{\bf Show your work}

\vspace{.25in}

%\nl{5}
{\large\bf NAME :\hbox to 12cm{\hrulefill}}

%\end{center}

\vspace{.25in}

{\bf 1 [12P])} Write the equation for the tangent line and the
normal line to the graph of 

a) ${\displaystyle f(x) = x^2 + 4 x +2 
}$
at the Point $P(1,7)$. \\

Tangent line:\hbox to 3.5cm{\hrulefill}\hfill
Normal line:\hbox to 3.5cm{\hrulefill}\hfill


b)  $\displaystyle{y = x \cos (x)}$
at the point $P(\frac{\pi}{2},0)$.
\\ %\nl{5}
Tangent line:\hbox to 3.5cm{\hrulefill}\hspace{1.5cm}
Normal line:\hbox to 3.5cm{\hrulefill}\hfill
\\ %\nlc{5}



\vspace{.25in}

{\bf 2 [12P])}  Find the limits

a) $\displaystyle{
\lim_{x\to 2}\frac{x^3 - 7x + 6}{x^2-x - 2}=
\hbox  to 3cm{\hrulefill}}$ 

\vspace{.25in}

b) ${\displaystyle \lim_{x\to 0}\frac{\sin 3x}{5x}=
\hbox to 3cm{\hrulefill}}$


c) ${\displaystyle \lim_{x\to \infty}\frac{x+\ln x}{x}=
\hbox to 3cm{\hrulefill}}$

\vspace{.25in}

{\bf 3 [14P])} Find the maximum and minumum values of
the function 

a) $\displaystyle{f(x) = x^2 - \frac{4}{x^2}}$ on the
closed interval $\displaystyle{[1,3]}$.
%\nl{5}
The maximum is:\hbox to 3cm{\hrulefill}\hspace{1.5cm}\hfill
The minimum value is: \hbox to 3cm{\hrulefill}
\\

b)
$\displaystyle{f(x) = |x^2-1| +\frac{1}{2}x^2}$ on the
closed interval $\displaystyle{[0,2]}$.
\\ %\nl{5}
The maximum is:\hbox to 2cm{\hrulefill}\hspace{1.5cm}
The minimum value is: \hbox to 2cm{\hrulefill}


\vspace{.25in}

{\bf 4 [18P])} Find the derivative of the following functions\\
%\nl{5}

a) ${\displaystyle f(x) = \frac{1}{x} + xe^{x^2}. \quad f^\prime (x) = }$
%\nlc{2}
\\

b) ${\displaystyle f(x) =  3^x + {\rm log}_{10}(x). \quad
f^\prime (x) = }$
%\nlc{2}
\\

c)  ${\displaystyle h(x) =  (x+1)^{-x} . \quad
h^\prime (x) = }$
%\nlc{2}


\vspace{.25in}

{\bf 5 [18P])}  Evaluate the following antiderivatives:
\\ %\nl{5}
a) 
$\displaystyle{\int \left( 4^x + x^{3/2} - \frac{1}{x^4}
\right)\, dx =}$
%\nlc{2} %\nl{5}
\\

b) $\displaystyle{\int \frac{1 + \ln (x)}{x} \, dx=}$

\vspace{.25in}

{\bf 6 [20P])} Evaluate the following integrals:
\\ %\nl{5}

a)  $\displaystyle{\int_{0}^3 (1 - 3x)^5\, dx =}$
%\nlc{3}
\\

b) ${ \int_0^{\pi /2} (\sin x)^2 \cos x\, dx\, = }$
\\

c) 
$\displaystyle{\int \left( x^5 + x^{3/2} - \frac{1}{x^4}
\right)\, dx =}$
\\ %\nl{5}

d) $\displaystyle{\int \tan x \ln |\cos x| \, dx=}$
\vspace{.25in}

{\bf 7 [12P])} Evaluate the sum $ 
\displaystyle{\sum_{j=1}^{20}\left( j^2\,-2j  +\, 1\right) =
\hbox  to 2cm{\hrulefill}}$

\vspace{.25in}%\nlc{3} 

{\bf 8 [24P])} Sketch the graph of the function
${\displaystyle f(x) = \frac{x^3}{x^2-1}}$. Determine the domain and
range. If there are any, then identify and label all
extrema ([4P]), inflection points ([4P]), intercepts
([4P]), and asymptotes ([4P]). Indicate the concave structure
clearly ([4P]).

\vspace{.25in}


{\bf 9 [12P])}
Use linear approximation to estimate
$\displaystyle{28^{2/3}\approx\hbox  to 3cm{\hrulefill} }$
%\newpage
%\nlc{2}
\vspace{.25in}

{\bf 10 [15P])} Sketch the region bounded by the curves
$\displaystyle{y=x^3}$ and ${\displaystyle y =  2x}$ and find its area.
\\
%\nl{4}

The area is : \hbox  to 3cm{\hrulefill}


\vspace{.25in}

{\bf 11 [15P])} Find the volume of the solid that is generated
by rotating the plane region bounded
by the curves  $\displaystyle{y= x}$ and
${\displaystyle y =  x^2}$  around the $y$-axes.
%\nl{4}
\\

The volume is: \hbox  to 3cm{\hrulefill} 
%\nlc{6}

\vspace{.25in}

{\bf 12 [15P])} Find the length of the arc
$\displaystyle{y = \frac{2}{3}\left(1 + x^2\right)^{3/2}}$
from $\displaystyle{x=0}$ to $\displaystyle{x = 2}$.
%\nl{4}
\\

The length is: \hbox  to 3cm{\hrulefill}
%\nlc{4}

\vspace{.25in}


{\bf 13 [14P])} Suppose that the fish population $P(t)$ in a
lake is given by the differential equation
${\displaystyle \frac{dP}{dt} = - k\sqrt{P(t)}}$. If
there were $576$ fish in the lake at $t=0$ and
$4$ weeks later only
$144$, how many are there after $5$ weeks and how long will it
take all the fish in the lake to die?

\vspace{.25in}


{\bf 14 [15P])} Find the volume of the solid that is generated
by rotating the plane region bounded
by the curves  $\displaystyle{y= x -x ^2,\quad
y = 0}$ around the axes  $y=-1$.
\\ %\nlc{6}

\vspace{.25in}

{\bf 15)} Lynda shoots an arrow straight upwards from the
ground with initial velocity $320$ ft/s.
\\ %\nl{5}
a) How high is the arrow after $3$ s?
\\
b) At what time is the arrow exactly $1200$ ft above
the ground?
\\
c) How many seconds after its release does the arrow strike the
ground?

\end{document}




\documentclass[12pt]{article}
\usepackage{amsmath,amssymb, amsfonts}
\usepackage{latexsym}
%\pagestyle{headings}
\pagestyle{empty}
%\input ../danp
%\catcode`\"=\active
\textwidth16cm
\textheight23cm
\oddsidemargin0cm
\evensidemargin-4.5mm
\topmargin-10mm

\renewcommand{\baselinestretch}{1.2}
\parindent0em

\begin{document}

\begin{large}
\begin{bf}
\hspace{1.5in}
{MATH 1550-24, Test 2} %. Fri. Mar.26, 2004}
\hspace{1in}
\parbox{1in}{ Name \newline Id}
\end{bf}
\end{large}

\vspace{0.2in}
%\noindent
%Use one page for each problem or sub-problem of the nine numbered questions(use %back of the page if necessary).

Put the question number (including its sub-problem number, if any) for each 
problem on your answer sheet. 
Put a box around the final answer to a question.

\vspace*{.02in}
For full credit you must \emph{show your work}. You must have enough
written work, including explanations when called for, to justify your
answers. Incomplete solutions may receive partial credit if you have
written down a reasonable partial solution. %Correct answers may not
                                            %receive credit if
                                         
\vspace*{.2in}

{\bf 1 [20P])}  Find the derivative of the following functions:
\\ 
a) $\displaystyle{f(x) = \sec (x)+\tan(x) \quad f^\prime(x) =}$

\vspace{.7in} 

b) $ \displaystyle{y = \frac{1  + \ln (x)}{1  - \ln (x)}. \quad dy/dx =}$ %\fbox{\rule[-7mm]{0cm}{1.5cm}\hbox to 8cm{}}

\vspace{.9in} 

%c) $\displaystyle{f(x) =  x^{\tan x} \quad f^\prime(x) =}$

%b) $\displaystyle{f(x) = \left(\frac{1}{x^2+1}\right)^2\quad f^\prime(x) =}$
%\fbox{\rule[-7mm]{0cm}{1.5cm}\hbox to 8cm{}}

c) $\displaystyle{y = \frac{e^x}{x^2-1}\quad dy/dx =}$

\vspace{.7in}

d) $\displaystyle{f(x) =  x^{e^x} \quad f^\prime(x) =}$

\vspace{.9in}

{\bf 2 [10P])} 
Find the slope of the tangent line to the graph
of the equation 
$\displaystyle{x^{2/3}+y^{2/3}=2}$
at the point $(1,1)$.
The slope is: %\fbox{\rule[-7mm]{0cm}{1.5cm}\hbox to 8cm{}}

\vspace{2.3in}

%{\bf 3 [5P])}  Explain if the equation $\displaystyle{x^3+x + 1}$ has a solution in the
%interval $[-1,0]$ or not.
 

{\bf 3 [12P])} Let $\displaystyle{f(x) = x^4-1}$.\\
a) Find the intervals on the $x$-axis on which the function is 
increasing as
well as those on which it is decreasing.
\\ 
b) Find the absolute maximum and absolute minimum on the interval
$[-5,5]$.
\\ 

%{\bf Bonus problem, 9P}: Determine for the following 3 functions the
%open intervals on the $x$-axis on which each function
%is increasing and those where it is decreasing. Then use
%this information to match the function to
%is graph:
%\\ %\nl{5}
%A) $\displaystyle{f(x) = xe^x }$,\hfill B) $\displaystyle{f(x) =
%\frac{1}{3}x^3+x^2+12}$
%\hfill
%C) $\displaystyle{f(x) = x^4-\frac{8}{3}x^3-2x^2+8x}$.
%\newpage

%{\bf 4 [6P])} Calculate the first two derivatives of the function
%$f(x) = x \ln x$.


%\hfill${\displaystyle f^\prime(x) = }$\hspace{5cm} ${\displaystyle f^{\prime\prime}(x) =
%}$\hspace{5cm}\hfill

\vspace{3in}

{\bf 4 [20P])} Sketch the graph of the function
${\displaystyle f(x) = \frac{x}{x- e}}$. Identify and label all
extrema ([4P]), inflection points ([4P]), intercepts
([4P]), and asymptotes ([4P]). Indicate the concave structure
clearly ([4P]).

\newpage

{\bf 5 [12P])}  Find the  
limits. Use the l'Hospital's Rule where applicable. If l'Hospital's Rule doesn't
apply, explain why.  
\\ 
a) $\displaystyle{\lim_{x\to -2} \frac{x+2}{x^3+2x^2} }$

\vspace{.5in}

b) $\displaystyle{\lim_{x\to \infty} \sqrt{x^2+x}-\sqrt{ x^2+17}
}$

\vspace{.5in}

c) $\displaystyle{\lim_{x\to \infty}  \frac{e^{2x}}{x^2}}$

\vspace{.5in}


d)  $\displaystyle{\lim_{x\to 0}  \frac{\tan^{-1} x}{x}}$

\vspace{2.5in}

%f)  $\displaystyle{\lim_{x\to 0}  (\cos x)^{1/x^2}}$


{\bf 6 [15P])}  Find the antiderivative of the following 
functions:
\\ %\nl{5}
a) $\displaystyle{\int {2x^3 - 3x^2} \, dx 
=}$

\vspace{.4in}

b) $\displaystyle{\int \frac{2}{x} \, dx=}$

\vspace{.4in}

c) $\displaystyle{\int  xe^{x^2}\, dx =}$
\\ %\nlc{1.5}
%d) $\displaystyle{\int x^5 + \frac{2}{\sqrt{x}}\, dx = }$
%\\ %\nlc{2}

%$\displaystyle{\int x^4 + \cos x - \frac{1}{x^{1/3}}\, dx}$.
%\\ %\nl{5}
%d) $\displaystyle{\int 2x^2\sin (x^3)\, dx}$.

%e) $\displaystyle{\int \frac{\ln (1/x)}{x}\, dx}$.

\vspace{3in}


%find the first 3 digits in $\displaystyle{\sqrt[3]{9}}$.
%{ Show each step in your calculations.}
%\\

\vspace{2in}

%b) Use linear approximation to extimate
%$\displaystyle{\frac{1}{15^{1/4}}}$.
%\\ %\nl{5}

{\bf 7 [11P])} Use Newton's method to find the solution (up to 4 decimal places 
of accuracy) to the equation $\displaystyle{x^3-3x= 0}$ in
the interval $\displaystyle{[1,2]}$.


%{\bf 10 [5P])} Find the maximum possible value of the product of two
%positive numbers whose sum is $50$.

\vspace{3in}

%{\bf 10 [8P]} A boat leaves a dock at 2:00 p.m. and travels due south at a speed of
%20 km/h. Another boat has been heading due east at 15 km/h and reaches the same dock at
%3:00 p.m. At what time were the two boats closest together?
  
\end{document}
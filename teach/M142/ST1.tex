\documentclass[12pt]{article}
\usepackage{amsmath,amssymb, amsfonts}
\usepackage{latexsym}
\pagestyle{empty}

\renewcommand{\baselinestretch}{1.2}
\parindent0em

\begin{document}

\begin{large}
\begin{bf}
\hspace{.75in}
{MATH 142 \quad Review Test 1} %. Thursday Feb.9, 2006}
\hspace{1in}
\parbox{1in}{ Name \newline Id}
\end{bf}
\end{large}

\vspace{0.2in}
%\noindent
%Use one page for each problem or sub-problem of the nine numbered questions(use %back of the page if necessary).

Put the question number (including its sub-problem number, if any) for each 
problem on your answer sheet. 
Put a box around the final answer to a question.

\vspace*{.02in}
For full credit you must \emph{show your work}. You must have enough
written work, including explanations when called for, to justify your
answers. Incomplete solutions may receive partial credit if you have
written down a reasonable partial solution. %Correct answers may not
                                            %receive credit if
                 

\vspace{.25in}

{\bf 1 [10P]} Evaluate the indefinite integrals
\begin{enumerate}
\item[a.] \; ${\displaystyle \int (x+3)^{5/2}dx }$.
\item[b.] \; ${\displaystyle \int \frac{x}{4 - x^2}\, dx \,  }$
\item[c.] \; ${\displaystyle \int \frac{1 + e^{2x}}{e^{x}}\, dx \, }$
%\item )\; ${\displaystyle \int \frac{1 + \ln x}{x}\, dx}$.
\item[d.] \; ${\displaystyle \int  \csc x\cot x dx} $
\end{enumerate}

%\newpage

{\bf 2.} Assume that ${\displaystyle [x_{i-1},x_i]}$ denotes the
$i^{th}$ subinterval of a subdivision of $[0,2]$,
which is divided into $n$ subintervals having the same length ${\displaystyle
\Delta x = \frac{2}{n}}$. \\
{\bf a [5P]} Evaluate the Riemann sum ${\displaystyle
\sum_{i=1}^n \left(x_i -0.5\right)\Delta x}$.

{\bf b [5P]} What is ${\displaystyle
\lim_{n\to \infty}
\sum_{i=1}^n \left(x_i -0.5\right)\Delta x}$.

{\bf c [5P]} Give an integral 
(but do not evaluate it) that is approximated by the Riemann sum in {\bf a}.

\vspace{.5in}

%\item ${\displaystyle \int x^2 \cos (x^3)\, dx }$. 
%\item ${\displaystyle \int \frac{({\rm ln}(x))^2}{x}\, dx \, = }$

%\newpage

{\bf 3 [10P]} Evaluate the definite integrals
\begin{enumerate}
\item[a.] \;${\displaystyle \int_e^{100} \frac{dx}{ x\ln x} }$.
%\item \;${\displaystyle \int_{-10}^{10} \frac{x}{x^2 + 1} \, dx }$
\item[b.] \;${\displaystyle \int^{1/2}_{-1/2} \frac{dt}{\sqrt{1-t^2}} }$.
%\item )\;${\displaystyle \int_0^{\pi / 2} \sin x \cos x\, dx}$.
%\item )\;${\displaystyle \int_{-1}^1 \frac{x}{x^2 + 2} \, dx }$.
\end{enumerate}

%\vspace{4.25in}

{\bf 4 [10P]} Find the derivative. 
{\bf a.} Use the Fundamental Theorem of Calculus to find:
\begin{equation*} 
\frac{d\ }{dx} ~ 
\int_{17}^x e^{\sqrt{t+5~~}} \, dt~ = ~
\end{equation*}

%\vspace{1.90in}
{\bf b.}
\begin{equation*} 
\frac{d\ }{dx} ~ 
7^{(\arctan x)^2} ~=~
\end{equation*}

{\bf 5 [10P]} Evaluate the integrals \quad
a. ${\displaystyl \int \tan x \sec^5 x dx}$.

%\vspace{2.67in}

b. ${\displaystyl \int x^2 \tan^{-1} x dx}$

%c. ${\displaystyl \int \frac{x^2}{ \sqrt{9- x^2}} dx}$

%d. ${\displaystyl \int \frac{ e^{\sqrt{x}}}{\sqrt{x}} dx}$

c. ${\displaystyl \int \frac{dx}{ x(2+\ln x)^2} } $

%f. ${\displaystyle \int \sec^3 x dx }$

d.  ${\displaystyle \int (\sec x+\tan x) dx }$

%h. ${\displaystyle \int (\cos^4 x+ \sin^4 x) dx }$

e. ${\displaystyle \int_0^1 \sin^{-1} x dx}$

%\newpage

{\bf 6 [10P]} Find the total area of the bounded region given by the
${\displaystyle x-}$axis and the graph of the function
${\displaystyle f(x) = x^3-x }$ on $[-1,1]$.

%\vspace{3.90in}
 
%{\bf 7 [12P]} Find the volume of the solid that is generated by the plane region bounded by
%${\displaystyle y = 9 - x^2 }$ and ${\displaystyle y = x^2 + 1}$ rotated
%around the ${\displaystyle x-}$axis.

%\vspace{.35in}

%{\bf 6 [15P]} Use either the method of disk/washer or the method of 
%cylindrical shells to express as an integral
%the volume of the solid generated by revolving around the
%$y$-axis the region bounded by $y= x^2$ and
%$y=\sqrt{x}$.  Do {\em not} evaluate the integral.

%{\bf 3 [12P]} Solve the initial value problem
%${\displaystyle \frac{dy}{dx} = \frac{1}{2yx},\quad
%y(1) =2}$.
%\\ %\nlc{3.5}

%{\bf 9 [11P]} Find the work done be the force ${\displaystyle
%F(x) = \frac{10}{x^2}}$
%in moving a particle along the $x$-axis from $x=1$ to $x=10$.

{\bf 7 [10P]} Find the work done by the force $
F(t) = \frac{12}{2t^2+1}$
in moving a particle along the $t$-axis from $t=0$ to $t=1.5$

%{\bf 10 [15P]} Use the method of cylindrical shells to find
%the volume of the solid generated by revolving around the
%$y$-axis the region bounded by $y= x^2$ and
%$y = 8 - x^2$.

%{\bf 8 [10P]} Let $R$ be the region enclosed by 
%\begin{equation*} 
%y = 9-x^2 \quad\textrm{and}\quad y = 0 ~. 
%\end{equation*} 
%Let $V$ be the volume of the solid obtained by 
%revolving the region $R$ about the $x$-axis.  

%{\bf a} Make a rough sketch of the region $R$, 
%labeling the important points.  

%{\bf b} Using the disk/washer method, 
%express the volume $V$ as an integral 
%(or maybe 2 integrals).  
%\newline You do NOT have to evaluate the integral(s). 

{\bf 8 [10P]} Let $R$ be the region enclosed by 
\begin{equation*} 
y = \sqrt{x}
\quad\textrm{and}\quad 
 x =4
\quad\textrm{and}\quad 
 x = 9
\quad\textrm{and}\quad
 y = 0 ~. 
\end{equation*} 
 Let $V$ be the volume of the solid obtained by 
revolving the region $R$ about the $y$-axis.  

%\vspace{3.2in}
{\bf a.}  Make a rough sketch of the region $R$, 
labeling the important points.  

{\bf b.}    Using the cylindrical shell  method, 
express the volume $V$ as an integral 
(or maybe 2 integrals).  
\newline 
You do NOT have to evaluate the integral(s). 

%\vspace{3.2in}
{\bf 9 [5P]} Let $L$ be the arc length of the curve 
\begin{equation*} 
x = \frac{y^4}{8} + \frac{y^{-2}}{4} 
\quad\textrm{from}\quad
y=1 
\quad\textrm{to}\quad
y=4 ~. 
\end{equation*} 
Express $L$ as an integral.  
You do NOT have to evaluate the integral.

%\vspace{3.2in}
{\bf 10 [10P]} Let $S$ be the area of the surface generated by 
revolving the curve below about the line $y=-0.5$
\begin{equation*} 
y = \frac{1}{x}\,,\quad \quad 2\le x
\le b, \;%\text{and}\;
\end{equation*} 
%{\bf a.} Make a rough sketch of the region $R$, 
%labeling the important points.  
where $b$ is a constant greater than 2.  Set up an integral for the surface area $S=S(b)$ 
and find its value.

%{$\displaystyle\int 
%\sec^3 x \tan^3 x \, dx ~=~$} 

%{$\displaystyle\int 
%\ln (1+x)  \, dx 
%~=~$} 

%{$\displaystyle\int 
%\cos^2 (3x)  \, dx ~=~$} 

%\vspace{3.3in}
%{$\displaystyle\int 
%\dfrac{dx}{\sqrt{x} ~ (1+x)}~=~$} 
% \noindent 
%Hint: $1+x = 1 + (\sqrt{x}~)^2$ 

% \vspace{3.3in}
%{$\displaystyle\int 
%x^2 e^{- 2x}  \, dx ~=~$} 

%\vspace{3.3in}
%{$\displaystyle\int 
%\frac{dx}{\sqrt{ 3+2x-x^2}}~=~$} 
% \noindent Hint: complete the square. 

%\vspace{3.3in}
%\mb{450}{50}{$\displaystyle\int 
%\frac{x^2}{x^2-3x+2}  \, dx ~=~$}  
 
\end{document}
\documentclass[12pt]{article}
\usepackage{amsmath, amssymb}
\usepackage{latexsym}

\topmargin = -.01in
\leftmargin = -.15in
\rightmargin =-.15in
\pagestyle{empty}

\begin{document}
\begin{center}
\begin{large}
{\bf Math 1550}\quad (Spring 2004)\\
\quad \\
 Analytic Geometry and Calculus I
\end{large}
\end{center}


%{\bf Course title:}  Analytic Geometry and Calculus I \\
\vspace{.1756in}
\noindent
{\bf Lectures:} MTWThF 8:40-9:30  (Section 24) in 134 Lockett Hall

\vspace{.1056in}
\noindent
{\bf Instructor:} Shijun Zheng\\  
{\bf Office:}  226 Lockett\\ 
{\bf Office hours:} 3-4:00 Tu and Thur, or by appointment\\
{\bf Phone:} (225) 578 1659 \\
{\bf E-mail:} szheng@math.lsu.edu\\
{\bf Course website:} http://www.math.lsu.edu$/^{\tiny{\sim}}$szheng/teach2/1550.html 

\vspace{.2in}
\noindent
{\bf Prerequisite:}
Math 1022 or 1023. 
 Credit will be given for only one of the following: Math 1431, 1441, 1550.
  
\vspace{.15in}
\noindent
{\bf Text:} Stewart, \emph{Calculus 5th. Early Transcendentals}, 5th edition (2003). 

\vspace{.2in}

\noindent{\bf Syllabus.}
We will cover most of Chapters 2-6 in Stewart. Time permitting, we may also cover
the first three sections of Chapter 8.  For the detailed 
syllabus, see the summary and schedule in my course website.

\vspace{0.15in}
\noindent
{\bf Grading.} 
The weighting in your course score  points (600 possible) will be 
$$
\begin{array}{ll}
200 \quad   & \text{Final Exam}\\          
100\; \text{each} \quad & \text{Three hour exams (including one Midterm)}\\    
  100   \quad & \text{Quizzes }
\end{array}
$$
  (Point totals will be normalized to these relative weights). 

\vspace{0.15in}
\noindent
{\bf Course Grades.} 
Your grade will be based on a point total.

A: 90$\%$ or above. \quad B: 80-89$\%$. \quad C: 70-79$\%$. \quad D: 60-69$\%$.\quad 
F: 59$\%$ or below.

\vspace{0.15in}
\noindent
{\bf Make-ups.} Makeups for exams will be given
only in extreme circumstances in
accordance with official University policy. Except for unforeseeable
events, requests for a makeup must be made
before the exam is given. 

**If you know BEFORE an exam that you have a conflict, 
contact me in advance. In this case, it is sometimes 
possible to arrange an early exam.**

\newpage
\vspace{0.19in}
\noindent
{\bf Philosophy.} 
Real learning requires your active involvement. 
Practice on solving the homework problems is strongly recommended.
The result in a test usually reflects how much effort you have put into 
the homework assignment as well as class learning, and how well you have prepared for it. 
%<The grading scheme rewards involvement and dedication, 
% by putting a lot of points on the section 
%work, homework and midterm exams-->  
%<The  midterm exams will not be surprises, they will 
%be closely related to the work you will have done.> 

%<font color="green">** Note especially, you really need to be careful to keep up 
%with the homework. Get into a groove **</font>
%  (It is a maxim in mathematics that >
% you really learn a subject when you have to teach it.)---> <br> 

\vspace{0.195in}
\noindent
{\bf  Math Tutoring.} \quad  9:30 - 5:30, Monday-Thursday. \quad
     9:30 - 3:00, Friday.\qquad Location:  39 Allen Hall 

\vspace{0.195in}
\noindent
{\bf Calculator.}
The use of calculator could be helpful (but not necessary)  for some homework and 
section quizzes.
  It is \emph{not} required to have
 knowledge of operating your calculator at an advanced level.
% A regular Texas Instruments graphics calculator (TI-82,83, ...)
%should be sufficient.>

% On the hour exams, however, graphing and programmable calculators 
%ill not be 
%llowed.  If you wish, you may use a cheap
%cientific calculator, but it should not be needed.>

\end{document}

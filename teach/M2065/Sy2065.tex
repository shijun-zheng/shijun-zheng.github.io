\documentclass[12pt]{article}
\usepackage{amsmath, amssymb}
\usepackage{latexsym}

\topmargin = -.01in
\leftmargin = -.15in
\rightmargin =-.15in
\pagestyle{empty}

\begin{document}
\begin{center}
\begin{large}
{\bf Math 2065}\quad (Fall 2004)\\
\quad \\
Elementary Differential Equations
\end{large}
\end{center}

%{\bf Course title:} Elementary Differential Equations\\
\vspace{.1756in}
\noindent
{\bf Lectures:} MWF 9:40-10:30 (Section 1) in 226 Tureaud Hall

\vspace{.1056in}
\noindent
{\bf Instructor:} Shijun Zheng\\  
{\bf Office:}  140 Lockett\\ 
{\bf Office hours:} 2-3:00 M and W, or by appointment\\
{\bf Phone:}  578 5329 \\
{\bf E-mail:} szheng@math.lsu.edu\\
{\bf Course website:} http://www.math.lsu.edu$/^{\tiny{\sim}}$szheng/teach2/2065.html 

\vspace{.2in}
\noindent
{\bf Prerequisite:}
Math 152.  
 Credit will be given for only one of the following: Math 2065, 2070, 2090.
  
\vspace{.15in}
\noindent
{\bf Text:} W.~Adkins and M.~Davidson, {\em Ordinary Differential Equations} (2004) 
\vspace{.162in}

\noindent{\bf Syllabus.}
We will cover most of Chapters 1-6 in the text. 
This is an introductory course in ordinary differential equations with emphasis on linear differential equations. A nuance of the present course is an early introduction of the Laplace transform into the theory, and then subsequently using the Laplace transform to extract much of the basic information about constant coefficient linear ODE in an expedited manner. 

%Time permitting, we may also cover
%the first three sections of Chapter 8.  For the detailed 
%syllabus, see the summary and schedule in my course website.

%\vspace{.152in}
%\noindent
%{\bf Exam Dates}:         
%$$
%\begin{array}{ll}
%Exam I \quad   & \text{Sept 22}\\          
%Exam II  \quad & \text{Oct 22}\\    
%Exam III \quad & \text{Nov 19}\\
%Final \quad  & \text{Dec 8 (10:00 - Noon)}
%\end{array}
%$$

\vspace{0.15in}
\noindent
{\bf Grading.} 
The weighting in your course score  points (650 possible) will be 
$$
\begin{array}{ll}
200 \quad   & \text{Final exam}\\          
100\; \text{each} \quad & \text{Three hour exams (including one midterm)}\\    
  150   \quad & \text{Quizzes/homework }
\end{array}
$$
  (Point totals will be normalized to these relative weights). 

\vspace{0.15in}
\noindent
{\bf Course Grades.} 
Your grade will be based on a point total.

A: 90$\%$ or above. \quad B: 80-89$\%$. \quad C: 70-79$\%$. \quad D: 60-69$\%$.\quad 
F: 59$\%$ or below.

\newpage
\vspace{0.215in}
\noindent
{\bf Make-ups.} Makeups for exams will be given
only in extreme circumstances in
accordance with official University policy. Except for unforeseeable
events, requests for a makeup must be made
before the exam is given. 

**If you know BEFORE an exam that you have a conflict, 
contact me in advance. In this case, it is sometimes 
possible to arrange an early exam.**

\vspace{0.19in}
\noindent
{\bf Philosophy.} 
Real learning requires your active involvement. 
Practice on solving the homework problems is strongly recommended.
The result in a test usually reflects how much effort you have put into 
the homework assignment as well as class learning, and how well you have prepared for it. 
%<The grading scheme rewards involvement and dedication, 
% by putting a lot of points on the section 
%work, homework and midterm exams-->  
%<The  midterm exams will not be surprises, they will 
%be closely related to the work you will have done.> 

%<font color="green">** Note especially, you really need to be careful to keep up 
%with the homework. Get into a groove **</font>
% you really learn a subject when you have to teach it.)---> <br> 

\vspace{0.195in}
\noindent
{\bf  Math Tutoring.} \quad  9:30$ - $5:30, Monday-Thursday. \quad
     9:30$ - $3:00, Friday.\qquad Location:  39 Allen Hall 

%\vspace{0.195in}
%\noindent
%{\bf Calculator.}
%The use of calculator could be helpful (but not necessary)  for some homework and 
%section quizzes.
%  It is \emph{not} required to have
% knowledge of operating your calculator at an advanced level.

% On the hour exams, however, graphing and programmable calculators 
%ill not be 
%llowed.  If you wish, you may use a cheap
%cientific calculator, but it should not be needed.>

\end{document}
